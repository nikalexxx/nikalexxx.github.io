% !TEX root =  all.tex
\section{Вполне упорядоченные множества}\index{Множества!вполне упорядоченные}
        \index{Вполне упорядоченные множества}%
        \label{well-ordered-sets}

Фундированные линейно упорядоченные множества называются
\emph{вполне упорядоченными}, а соответствующие порядки\т \emph{полными}.
        \index{Полный порядок}%
        \index{Отношение!полного порядка}%
Для линейных порядков понятия наименьшего и минимального
элемента совпадают, так что во вполне упорядоченном множестве
всякое непустое подмножество имеет наименьший элемент.

Заметим, что частично упорядоченное множество, в котором всякое
непустое подмножество имеет наименьший элемент, автоматически
является линейно упорядоченным (в самом деле, всякое
двухэлементное множество имеет наименьший элемент, поэтому любые
два элемента сравнимы).

Примеры вполне упорядоченных множеств: $\bbN$,
$\bbN\hm+k$ (здесь $k$~обозначает конечное линейно
упорядоченное множество из $k$~элементов),
$\bbN\hm+\bbN$,~$\bbN\hm\times\bbN$.

Наша цель\т понять, как могут быть устроены вполне
упорядоченные множества. Начнём с нескольких простых
замечаний.

\begin{itemize}
        %
\item
Вполне упорядоченное множество имеет наименьший элемент.
(Непосредственное следствие определения.)
        %
\item
Для каждого элемента~$x$ вполне упорядоченного множества (кроме
наибольшего) есть непосредственно следующий\index{Непосредственно
следующий элемент} за ним элемент~$y$
(это означает, что~$y\hm>x$, но не существует~$z$, для
которого~$y\hm>z\hm>x$). В самом деле, если множество всех элементов,
больших~$x$, непусто, то в нём есть минимальный элемент~$y$,
который и будет искомым. Такой элемент обычно обозначают~$x\hm+1$,
следующий за ним\т $x+2$~\итд
        %
\item
Некоторые элементы вполне упорядоченного множества могут не иметь
непосредственно предыдущего. Например, в
множестве~$\mathbb{N}\hm+\mathbb{N}$ есть два элемента, не имеющих
непосредственно предыдущего (наименьший элемент, а также
наименьший элемент второй копии натурального ряда). Такие
элементы называют \emph{предельными}\index{Предельный элемент}.
        %
\item\label{limit-natural}
Всякий элемент упорядоченного множества имеет вид~$z\hm+n$, где
$z$\т предельный, а $n$\т натуральное число (обозначение~$z\hm+n$
понимается в описанном выше смысле). В самом деле, если $z$~не
предельный, возьмём предыдущий, если и он непредельный\т
то его предыдущий \итд, пока не дойдём до предельного
(бесконечно продолжаться это не может, так как множество вполне
упорядочено). Очевидно, такое представление однозначно
(у элемента может быть только один непосредственно предыдущий).
        %
\item
Любое ограниченное сверху множество элементов вполне упорядоченного
множества имеет точную верхнюю грань.
(Как обычно, подмножество~$X$ частично
упорядоченного множества~$A$ называется \emph{ограниченным сверху}%
\index{Множества!ограниченные сверху},
если оно имеет \emph{верхнюю границу}\index{Верхняя граница},
\те элемент~$a\hm\in A$,
для которого $x\hm\le a$ при всех~$x\hm\in X$. Если среди
всех верхних границ данного подмножества есть наименьшая, то она
называется \emph{точной верхней гранью}\index{Точная верхняя грань}.)

В самом деле, множество всех верхних границ непусто и потому
имеет наименьший элемент. (Заметим в скобках, что вопрос о
точной нижней грани для вполне упорядоченного множества
тривиален, так как всякое множество имеет наименьший элемент.)
        %
\end{itemize}

Пусть $A$\т произвольное вполне упорядоченное множество. Его
наименьший элемент обозначим через~$0$. Следующий за ним элемент
обозначим через~$1$, следующий за~$1$\т через~$2$ \итд Если
множество конечно, процесс этот оборвётся. Если бесконечно,
посмотрим, исчерпали ли мы все элементы множества~$A$. Если нет,
возьмём минимальный элемент из оставшихся. Обозначим его~$\omega$%
\index{$\omega$}.
Следующий за ним элемент (если он есть) обозначим~$\omega\hm+1$,
затем $\omega\hm+2$~\итд Если и на этом множество не
исчерпается, то возьмём наименьший элемент из оставшихся,
назовём его~$\omega\hm\cdot2$\index{$\omega\cdot 2\ $},
и повторим всю процедуру. Затем
будут $\omega\hm\cdot3$, $\omega\hm\cdot4$~\итд Если и на этом множество не
кончится, минимальный из оставшихся элементов назовём~$\omega^2$%
\index{$\omega^2$}. Затем
пойдут $\omega^2\hm+1$, $\omega^2\hm+2$, $\dots$, $\omega^2\hm+\omega$,
$\dots$, $\omega^2\hm+\omega\hm\cdot2$,~$\dots$,
$\omega^2\hm\cdot2$,~$\dots$, $\omega^2\hm\cdot3$, $\dots$, $\omega^3$,
$\dots$ (мы не поясняем сейчас подробно обозначения).

Что, собственно говоря, доказывает это рассуждение? Попытаемся
выделить некоторые утверждения. При
этом полезно
такое определение: если линейно упорядоченное множество~$A$
разбито на две (непересекающиеся) части $B$ и~$C$, причём любой
элемент~$B$ меньше любого элемента~$C$, то $B$~называют
\emph{начальным отрезком}\index{Начальный отрезок} множества~$A$.
Другими словами,
подмножество~$B$ линейно упорядоченного множества~$A$ является
начальным отрезком, если любой элемент~$B$ меньше любого
элемента~$A\hm\setminus B$. Ещё одна переформулировка: $B\hm\subset A$
является начальным отрезком, если из $a,b\hm\in A$, \;$b\hm\in B$
и~$a\hm\le b$ следует~$a\hm\in B$. Заметим, что начальный отрезок
может быть пустым или совпадать со всем множеством.

Отметим сразу же несколько простых свойств начальных
отрезков:

\begin{itemize}
        %
\item
Начальный отрезок вполне упорядоченного множества
(как, впрочем, и любое подмножество) является
вполне упорядоченным множеством.
        %
\item
Начальный отрезок начального отрезка есть начальный отрезок
исходного множества.
        %
\item
Объединение любого семейства начальных отрезков (в одном и том
же упорядоченном множестве) есть начальный отрезок того же
множества.
        %
\item
Если $x$\т произвольный элемент вполне упорядоченного
множества~$A$, то множества~$[0,x)$ (все элементы множества~$A$,
меньшие~$x$) и~$[0,x]$ (элементы множества~$A$, меньшие или
равные~$x$) являются начальными отрезками.
        %
\item\label{initial-segment-structure}%
Всякий начальный отрезок~$I$ вполне упорядоченного
множества~$A$, не совпадающий со всем множеством, имеет вид~$[0,x)$
для некоторого~$x\hm\in A$. (В~самом деле, если~$I\hm\ne
A$, возьмём наименьший элемент~$x$ в множестве~$A\hm\setminus I$.
Тогда все меньшие элементы принадлежат~$I$, сам~$x$ не
принадлежит~$I$ и все б\'ольшие~$x$ элементы не принадлежат~$I$,
иначе получилось бы противоречие с определением начального
отрезка.)
        %
\item
Любые два начальных отрезка вполне упорядоченного множества
сравнимы по включению, \те
один есть подмножество другого.
(Следует из пре\-ды\-ду\-щего.)
        %
\item
Начальные отрезки вполне упорядоченного множества~$A$,
упорядоченные по включению, образуют вполне упорядоченное
множество. Это множество состоит из наибольшего элемента
(всё~$A$) и остальной части, изоморфной множеству~$A$. (В самом
деле, начальные отрезки множества~$A$, не совпадающие с~$A$,
имеют вид~$[0,x)$, и соответствие~$[0,x)\hm\leftrightarrow x$ будет
изоморфизмом.)
        %
\end{itemize}

Возвратимся к нашему рассуждению с последовательным выделением
различных элементов из вполне упорядоченного множества. Его
первую часть можно считать доказательством такого утверждения:
если вполне упорядоченное множество бесконечно, то оно имеет
начальный отрезок, изоморфный~$\omega$. (Говоря о множестве
натуральных чисел вместе с порядком, обычно употребляют
обозначение~$\omega$\index{$\omega$}, а не~$\bbN$.)

Но на этом наше рассуждение не оканчивается. Его следующая часть
может считаться доказательством такого факта: либо $A$~изоморфно
некоторому начальному отрезку множества~$\omega^2$, либо оно
имеет начальный отрезок, изоморфный~$\omega^2$. (Здесь
$\omega^2$\index{$\omega^2$}\т вполне упорядоченное множество пар натуральных
чисел: сравниваются сначала вторые компоненты пар, а при их
равенстве\т первые.)

Вообще верно такое утверждение: для любых двух вполне
упорядоченных множеств одно изоморфно начальному отрезку
другого, и доказательство состоит более или менее в повторении
проведённого рассуждения. Но чтобы сделать это аккуратно, нужна
некоторая подготовка.

\section{Трансфинитная индукция}\index{Трансфинитная индукция (рекурсия)}
        \label{transfinite-recursion}

Термины \лк индукция\пк\ и \лк рекурсия\пк\ часто употребляются
вперемежку. Например, определение факториала
$n!=1\cdot2\cdot3\cdot\ldots\cdot n$ как функции~$f(n)$, для
которой $f(n)\hm=n\cdot f(n-1)$ при~$n\hm>0$ и $f(0)\hm=1$,
называют и \лк индуктивным\пк, и \лк рекурсивным\пк. Мы будем
стараться разграничивать эти слова так: если речь идёт о
доказательстве чего\д то сначала для~$n\hm=0$, затем для
$n\hm=1,2,\dots$, причём каждое утверждение опирается на
предыдущее, то это \emph{индукция}\index{Индукция}. Если же мы определяем что\д то
сначала для~$n\hm=0$, потом для $n\hm=1,2,\dots$, причём
определение каждого нового значения использует ранее
определённые, то это \emph{рекурсия}.

Наша цель\т научиться проводить индуктивные доказательства и
давать рекурсивные определения не только для натуральных чисел, но и для других вполне упорядоченных множеств (эту технику иногда называют <<трансфинитной индукцией>>).

Доказательства по индукции мы уже обсуждали, говоря о
фундированных множествах (см.~раздел~\ref{well-founded-sets}), и
сейчас ограничимся только одним примером.

\begin{theorem}\label{monotone-mappings-wosets}
        %
Пусть множество $A$ вполне упорядочено, а отображение $f\colon
A\hm\to A$ возрастает (то есть $f(x)\hm<f(y)$ при~$x\hm<y$).
Тогда $f(x)\hm\ge x$ для всех~$x\hm\in A$.
        %
\end{theorem}

\begin{proof}
        %
Согласно принципу индукции (теорема~\ref{well-founded-induction},
с.~\pageref{well-founded-induction})
достаточно доказать неравенство $f(x)\hm\ge x$, предполагая, что
$f(y)\hm\ge y$ при всех~$y\hm<x$. Пусть это не так и~$f(x)\hm<x$. Тогда
по монотонности~$f(f(x))\hm<f(x)$. Но, с другой стороны, элемент~$y\hm=f(x)$
меньше~$x$, и потому по предположению индукции $f(y)\hm\ge y$, то
есть~$f(f(x))\hm\ge f(x)$.

Если угодно, можно в явном виде воспользоваться существованием
наименьшего элемента и изложить это же рассуждение так. Пусть
утверждение теоремы неверно. Возьмём наименьшее~$x$, для
которого~$f(x)\hm<x$. Но тогда $f(f(x))\hm<f(x)$ по монотонности и
потому $x$~не является наименьшим вопреки предположению.

Наконец, это рассуждение можно пересказать и так:
если $x\hm>f(x)$, то по монотонности
        $$
x > f(x) > f(f(x)) > f(f(f(x))) > \ldots,
        $$
но бесконечных убывающих последовательностей в фундированном
множестве быть не может.
        %
\end{proof}

Теперь перейдём к рекурсии. В определении факториала $f(n)$~выражалось
через~$f(n-1)$. В общей ситуации значение~$f(n)$
может использовать не только одно предыдущее значение функции,
но и все значения на меньших аргументах. Например, можно
определить функцию $f\colon\bbN\hm\to\bbN$, сказав, что
$f(n)$~на единицу больше суммы всех предыдущих значений, то
есть $f(n)\hm=f(0)\hm+f(1)\hm+\ldots\hm+f(n-1)\hm+1$; это вполне законное
рекурсивное определение (надо только пояснить, что пустая
сумма считается равной нулю, так что~$f(0)\hm=1$).

\begin{problem}
        %
Какую функцию~$f$ задаёт такое определение?
        %
\end{problem}

\problskip
Как обобщить эту схему на произвольные вполне упорядоченные
множества вместо натурального ряда? Пусть $A$~вполне
упорядочено. Мы хотим дать рекурсивное определение%
\index{Рекурсивное определение} функции $f\colon A\hm\to B$
(где~$B$\т некоторое множество). Такое определение должно
связывать значение~$f(x)$ на некотором элементе~$x\hm\in A$ со
значениями~$f(y)$ при всех~$y\hm<x$. Другими словами,
рекурсивное определение указывает~$f(x)$, предполагая известным
ограничение функции~$f$ на начальный отрезок~$[0,x)$. Вот точная
формулировка:

\begin{theorem}\label{transfinite-recursion-theorem}
        %
Пусть $A$\т вполне упорядоченное множество. Пусть $B$\т
произвольное множество. Пусть имеется некоторое рекурсивное
правило, то есть отображение~$F$, которое ставит в соответствие
элементу~$x\hm\in A$ и функции $g\colon [0,x)\hm\to B$ некоторый элемент
множества~$B$. Тогда существует и единственна функция $f\colon A\hm\to B$,
для которой
        %
        $$
f(x)= F(x,f|_{[0,x)})
        $$
при всех $x\hm\in A$. (Здесь $f|_{[0,x)}$\index{$f\vert_{[0,x)}$} обозначает
ограничение\index{Ограничение функции}\index{Функция!ограничение}
функции~$f$ на начальный отрезок~$[0,x)$\т мы отбрасываем все
значения функции на элементах, больших или равных~$x$.)
        %
\end{theorem}

\begin{proof}
        %
Неформально можно рассуждать так: значение~$f$ на минимальном
элементе определено однозначно, так как предыдущих значений
нет (сужение~$f|_{[0,0)}$ пусто). Тогда и на следующем
элементе значение функции~$f$~определено однозначно, поскольку на предыдущих
(точнее, единственном предыдущем) функция~$f$ уже задана, \итд

Конечно, это надо аккуратно выразить формально. Вот как это
делается. Докажем по индукции такое утверждение о произвольном
элементе~$a\hm\in A$:
        %
\quot{%
        %
существует и единственно отображение~$f$ отрезка~$[0,a]$ в
множество~$B$, для которого рекурсивное определение
\textup(равенство, приведённое в условии\textup) выполнено при
всех~$x\hm\in [0,a]$.
        }
\noindent
Будем называть отображение $f\colon [0,a]\hm\to B$, обладающее
указанным свойством, \emph{корректным}. Таким образом,
мы хотим доказать, что для каждого~$a\hm\in A$ есть единственное
корректное отображение отрезка~$[0,a]$ в~$B$.

Поскольку мы рассуждаем по индукции, можно предполагать, что для
всех $c<a$ это утверждение выполнено, то есть существует и
единственно корректное отображение $f_c\colon [0,c]\hm\to B$.
(Корректность~$f_c$ означает, что при всех~$d\hm\le c$ значение~$f_c(d)$
совпадает с предписанным по рекурсивному правилу.)

Рассмотрим отображения $f_{c_1}$ и $f_{c_2}$ для двух различных
$c_1$ и~$c_2$. Пусть, например,~$c_1\hm<c_2$. Отображение~$f_{c_2}$
определено на большем отрезке~$[0,c_2]$. Если ограничить~$f_{c_2}$ на
меньший отрезок~$[0,c_1]$, то оно совпадёт с~$f_{c_1}$,
поскольку ограничение корректного отображения на меньший
отрезок корректно (это очевидно), а мы предполагали
единственность на отрезке~$[0,c_1]$.

Таким образом, все отображения~$f_c$ согласованы друг с другом,
то есть принимают одинаковое значение, если определены
одновременно. Объединив их, мы получаем некоторое единое
отображение~$h$, определённое на~$[0,a)$. Применив к~$a$ и~$h$
рекурсивное правило, получим некоторое значение~$b\hm\in B$.
Доопределим~$h$ в точке~$a$, положив~$h(a)\hm=b$.
Получится отображение $h\colon [0,a]\hm\to B$; легко
понять, что оно корректно.

Чтобы завершить индуктивный переход, надо проверить, что на отрезке~$[0,a]$
корректное отображение единственно. В самом деле, его
ограничения на отрезки~$[0,c]$ при~$c\hm<a$ должны совпадать с~$f_c$,
поэтому осталось проверить однозначность в точке~$a$\т
что гарантируется рекурсивным определением (выражающим значение
в точке~$a$ через предыдущие).
        %
На этом индуктивное доказательство заканчивается.

Осталось лишь заметить, что для разных~$a$ корректные
отображения отрезков~$[0,a]$ согласованы друг с другом (сужение
корректного отображения на меньший отрезок корректно,
применяем единственность) и потому вместе задают некоторую
функцию $f\colon A\hm\to B$, удовлетворяющую рекурсивному определению.

Существование доказано; единственность тоже понятна, так
как ограничение этой функции на любой отрезок~$[0,a]$
корректно и потому однозначно определено, как мы видели.
        %
\end{proof}

Прежде чем применить эту теорему и доказать, что из двух
вполне упорядоченных множеств одно является отрезком другого,
нам потребуется её немного усовершенствовать.
Нам надо предусмотреть ситуацию, когда рекурсивное правило
не всюду определено. Пусть, например, мы определяем последовательность
действительных чисел соотношением~$x_{n}\hm=\tg x_{n-1}$ и начальным
условием~$x_0\hm=a$. При некоторых значениях~$a$ может оказаться, что построение
последовательности обрывается, поскольку тангенс не определён
для соответствующего аргумента.

\begin{problem}
        %
Докажите, что множество всех таких \лк исключительных\пк~$a$
(когда последовательность конечна) счётно.
        %
\end{problem}

\problskip
Аналогичная ситуация возможна и для общего случая.

\begin{theorem}\label{partial-transfinite-recursion}
        %
Пусть отображение $F$, о котором шла речь в
теореме~\ref{transfinite-recursion-theorem}, является частичным
(для некоторых элементов~$x$ и функций $g\colon [0,x)\hm\to B$
оно может быть не определено). Тогда существует функция~$f$,
которая
        %
\begin{itemize}
        %
\item
либо определена на всём~$A$ и согласована с рекурсивным
определением;
        %
\item
либо определена на некотором начальном отрезке~$[0,a)$ и
на нём согласована с рекурсивным определением, причём
для точки~$a$ и функции~$f$ рекурсивное правило неприменимо
(отображение~$F$ не определено).
        %
\end{itemize}
        %
\end{theorem}

\begin{proof}
        %
Это утверждение является обобщением, но одновременно и
следствием предыдущей теоремы~\ref{transfinite-recursion-theorem}. В
самом деле, добавим к множеству~$B$ специальный элемент~$\bot$
(\лк неопределённость\пк) и модифицируем рекурсивное правило:
новое правило даёт значение~$\bot$, когда старое было не
определено. (Если среди значений функции на предыдущих
аргументах уже встречалось~$\bot$, новое рекурсивное правило тоже
даёт~$\bot$.)

Применив теорему~\ref{transfinite-recursion-theorem} к
модифицированному правилу, получим некоторую функцию~$f'$. Если
эта функция нигде не принимает значения~$\bot$, то реализуется
первая из двух возможностей, указанных в теореме
(при~$f\hm=f'$). Если же функция~$f'$ принимает значение~$\bot$
в какой\д то точке, то она имеет то же значение~$\bot$ и во всех
больших точках. Заменив значение~$\bot$ на неопределённость, мы
получаем из функции~$f'$ функцию~$f$. Область определения
функции~$f$ есть некоторый начальный отрезок~$[0,a)$ и
реализуется вторая возможность, указанная в формулировке
теоремы.
        %
\end{proof}

\begin{problem}
        %
Сформулируйте и докажите утверждение об однозначности
функции, заданной частичным рекурсивным правилом.
        %
\end{problem}

Теперь у нас всё готово для доказательства теоремы о
сравнении вполне упорядоченных множеств.
     \index{Вполне упорядоченные множества!сравнение}%

\begin{theorem}\label{comparing-ordinals}
        %
Пусть $A$ и $B$\т два вполне упорядоченных множества. Тогда либо
$A$~изоморфно некоторому начальному отрезку множества~$B$, либо
$B$~изоморфно некоторому начальному отрезку множества~$A$.
        %
\end{theorem}

\begin{proof}
        %
Отметим прежде всего, что начальный отрезок может совпадать со
всем множеством, так что случай изоморфных множеств~$A$ и~$B$
также покрывается этой теоремой.

Определим отображение~$f$ из~$A$ в~$B$ таким рекурсивным
правилом: для любого~$a\hm\in A$
        %
\quot{%
        %
$f(a)$~есть наименьший элемент множества~$B$,
который не встречается
среди~$f(a')$ при~$a'\hm<a$.
        }

Это правило не определено в том случае, когда значения~$f(a')$
при~$a'\hm<a$ покрывают всё~$B$. Применяя
теорему~\ref{partial-transfinite-recursion}, мы получаем функцию~$f$,
согласованную с этим правилом. Теперь рассмотрим два случая:

\begin{itemize}
        %
\item
Функция~$f$ определена на всём~$A$.
Заметим, что рекурсивное определение гарантирует монотонность,
поскольку $f(a)$~определяется как минимальный ещё не использованный
элемент; чем больше~$a$, тем меньше остаётся
неиспользованных элементов
и по\-то\-му минимальный элемент
может только возрасти (из определения следует
также, что одинаковых значений быть не может).
Остаётся лишь проверить, что множество значений функции~$f$, то
есть~$f(A)$, будет начальным отрезком. В самом деле, пусть~$b\hm<f(a)$
для некоторого~$a\hm\in A$; надо проверить, что $b$~также
является значением функции~$f$. Действительно, согласно
рекурсивному определению $f(a)$~является наименьшим неиспользованным
значением, следовательно, $b$~уже использовано, то есть встречается
среди~$f(a')$ при~$a'\hm<a$.
        %
\item
Функция~$f$ определена лишь на некотором
начальном отрезке~$[0,a)$. В этом случае этот начальный отрезок
изоморфен~$B$, и функция~$f$ является искомым изоморфизмом.
В самом деле, раз $f(a)$~не определено, то среди значений
функции~$f$ встречаются все элементы множества~$B$.
С другой стороны, $f$~сохраняет порядок в силу рекурсивного
определения.
        %
\end{itemize}
        %
Таким образом, в обоих случаях утверждение
теоремы верно.
        %
\end{proof}

Может ли быть так, что $A$~изоморфно начальному отрезку~$B$, а
$B$~изоморфно начальному отрезку $A$? Нет\т
за исключением
тривиального случая, когда начальные отрезки представляют собой
сами множества~$A$ и~$B$. Это вытекает из такого утверждения:
        %
\begin{theorem}\label{nontrivial-initial-segment}
        %
Никакое вполне упорядоченное множество не изоморфно своему
начальному отрезку\index{Начальный отрезок} (не совпадающему
со всем множеством).
        %
\end{theorem}
        %
\begin{proof}
        %
Пусть вполне упорядоченное множество~$A$ изоморфно своему
начальному отрезку, не совпадающему со всем множеством. Как мы
видели на с.~\pageref{initial-segment-structure},
этот отрезок имеет вид~$[0,a)$ для некоторого
элемента~$a\hm\in A$. Пусть $f\colon A\hm\to[0,a)$\т изоморфизм. Тогда
$f$~строго возрастает, и по теореме~\ref{monotone-mappings-wosets}
имеет место неравенство~$f(a)\ge a$, что противоречит тому, что
множество значений функции~$f$ есть~$[0,a)$.
        %
\end{proof}

Если множество~$A$~изоморфно начальному отрезку
множества~$B$, а множество~$B$~изоморфно начальному отрезку
множества~$A$, то композиция этих изоморфизмов даёт изоморфизм
между множеством~$A$ и его начальным отрезком (начальный отрезок
начального отрезка есть начальный отрезок). Этот начальный
отрезок обязан совпадать со всем множеством~$A$, так что это
возможно лишь если $A$ и~$B$ изоморфны.

Сказанное позволяет сравнивать
вполне упорядоченные множества.
Если $A$~изоморфно начальному отрезку множества~$B$, не
совпадающему со всем~$B$, то говорят, что \emph{порядковый тип
множества~$A$ меньше порядкового типа множества~$B$}.
Если
множества $A$ и $B$~изоморфны, то говорят, что у них
\emph{одинаковые порядковые типы}. Наконец, если $B$~изоморфно
начальному отрезку множества~$A$, то говорят, что \emph{порядковый
тип множества~$A$ больше порядкового типа множества~$B$}.
Мы только что доказали такое утверждение:
        %
\begin{theorem}\label{ordinals-trichotomy}
        %
Для любых вполне упорядоченных множеств $A$ и~$B$
имеет место ровно один из указанных трёх случаев.
        %
\end{theorem}

\begin{problem}
     \label{subset-initial}
Пусть $A$\т вполне упорядоченное множество, а $B$\т его подмножество с индуцированным порядком (и, тем самым, тоже вполне упорядоченное множество). Покажите, что $B$ изоморфно начальному отрезку $A$. Приведите пример, когда этот начальный отрезок совпадает со всем множеством $A$, хотя $B\ne A$.  (Указание. Если $A$ изоморфно собственному начальному отрезку множества $B$, нарушается теорема~\ref{monotone-mappings-wosets}.)
     %
\end{problem}


Если временно забыть о проблемах оснований теории множеств и
определить порядковый тип\index{Порядковый тип} упорядоченного множества как класс
изоморфных ему упорядоченных
множеств, то можно сказать, что мы
определили линейный порядок на порядковых типах вполне
упорядоченных множеств (на \emph{ординалах}\index{Ординалы}, как говорят).
Этот порядок будет полным. Мы переформулируем это утверждение так,
чтобы избегать упоминания классов.
        %
\begin{theorem}\label{complete-order-on-ordinals}
        %
Всякое непустое семейство вполне упорядоченных множеств
имеет \лк наименьший элемент\пк\т множество, изоморфное
начальным отрезкам всех остальных множеств.
        %
\end{theorem}

\begin{proof}
        %
Возьмём какое\д то множество~$X$ семейства. Если оно наименьшее, то
всё доказано. Если нет, рассмотрим все множества семейства,
которые меньше его, то есть изоморфны его начальным отрезкам
вида~$[0,x)$. Среди всех таких элементов~$x$ выберем наименьший.
Тогда соответствующее ему множество и будет наименьшим.
        %
\end{proof}

\begin{problem}
     %
Покажите, что для любого вполне упорядоченного множества $A$ существует равномощное ему вполне упорядоченное множество $B$ с таким свойством: любой начальный отрезок $B$ (кроме всего $B$) имеет меньшую мощность, чем $B$.  (Множества с этим свойством\т точнее, их порядковые типы\т называют \emph{кардиналами}.)\index{Кардинал}
     %
\end{problem}

Из теоремы~\ref{ordinals-trichotomy} следует, что любые два вполне
упорядоченных множества сравнимы по мощности (одно равномощно
подмножеству другого). Сейчас мы увидим, что всякое множество
может быть вполне упорядочено (теорема~Цермело), и, следовательно,
любые два множества сравнимы по мощности.

\section{Теорема Цермело}\index{Теорема!Цермело}
        \label{zermelo}

\begin{theorem}[Цермело]\glossary{Цермело@\Цермело}
        \label{theorem-zermelo}
Всякое множество может быть вполне упорядочено.
        %
\end{theorem}

\begin{proof}
        %
Доказательство этой теоремы существенно использует аксиому
выбора\index{Аксиома!выбора} и вызывало большие нарекания
своей неконструктивностью.
На счётных множествах полный порядок указать легко
(перенеся с~$\bbN$). Но уже на множестве действительных
чисел никакого конкретного полного порядка указать не удаётся, и
доказав (с помощью аксиомы выбора) его существование, мы так и
не можем себе этот порядок представить.

Объясним, в какой форме используется аксиома
выбора. Пусть $A$\т данное нам множество. Мы принимаем, что
существует функция~$\varphi$, определённая на всех подмножествах
множества~$A$, кроме самог\'о~$A$, которая указывает
один из элементов вне этого подмножества:
        $$
X\subsetneq A \Rightarrow \varphi(X)\in A\setminus X.
        $$

После того как такая функция фиксирована, можно построить
полный порядок на~$A$, и в этом построении уже нет никакой
неоднозначности. Вот как это делается.

Наименьшим элементом множества~$A$ мы объявим
элемент~$a_0\hm=\varphi(\varnothing)$. За ним идёт
элемент~$a_1\hm=\varphi(\{a_0\})$; по построению он отличается от~$a_0$.
Далее следует
элемент~$a_2\hm=\varphi(\{a_0,a_1\})$. Если
множество~$A$ бесконечно, то такой процесс можно продолжать и
получить последовательность $\{a_0,a_1,\dots\}$ элементов
множества~$A$. Если после этого остаются ещё не использованные
элементы множества~$A$, рассмотрим элемент
$a_\omega\hm=\varphi(\{a_0,a_1,a_2,\dots\})$ и так будем продолжать,
пока всё~$A$ не кончится; когда оно кончится, порядок выбора
элементов и будет полным порядком на~$A$.

Конечно, последняя фраза нуждается в уточнении\т что значит \лк
так будем продолжать\пк? Возникает желание применить теорему о
трансфинитной рекурсии (у нас очень похожая ситуация: следующий
элемент определяется рекурсивно, если известны все предыдущие).
И это можно сделать, если у нас есть другое вполне упорядоченное
множество~$B$, и получить взаимно однозначное соответствие
либо между~$A$ и частью~$B$, либо между~$B$ и частью~$A$. В
первом случае всё хорошо, но для этого надо иметь вполне
упорядоченное множество~$B$ по крайней мере той же мощности, что
и~$A$, так что получается некий порочный круг.

Тем не менее из него можно выйти. Мы сделаем это так:
рассмотрим все потенциальные кусочки будущего порядка и
убедимся, что их можно склеить.

Пусть $(S,\le_S)$\т некоторое подмножество множества~$A$
и заданный на нём порядок. Будем говорить, что
$(S,\le_S)$~является \emph{корректным фрагментом}, если оно является
вполне упорядоченным множеством, причём
        $
s = \varphi ([0,s))
        $
для любого~$s\hm\in S$. Здесь $[0,s)$\т начальный отрезок
множества~$S$, состоящий из всех элементов, меньших~$s$
с точки зрения заданного на~$S$ порядка.

Например, множество $\{\varphi(\varnothing)\}$ является
корректным фрагментом (порядок здесь можно не указывать, так как
элемент всего один). Множество $\{\varphi(\varnothing),
\varphi(\{\varphi(\varnothing)\})\}$ (первый из выписанных
элементов считается меньшим второго) также является корректным
фрагментом. Это построение можно продолжать и дальше, но нам
надо каким\д то образом \лк перескочить\пк\ через бесконечное (и
очень большое в смысле мощности) число шагов этой конструкции.

План такой: мы докажем, что любые два корректных фрагмента
в определённом смысле согласованы, после чего
рассмотреть объединение всех корректных фрагментов. Оно
будет корректным и будет совпадать со всем множеством~$A$
(в противном случае его можно было бы расширить и получить
корректный фрагмент, не вошедший в объединение).

\textsf{Лемма 1.} Пусть $(S,\le_S)$ и $(T,\le_T)$\т два корректных
фрагмента. Тогда один из них является начальным отрезком
другого, причём порядки согласованы (два общих элемента
всё равно как сравнивать\т в смысле~$\le_S$ или в смысле~$\le_T$).

Заметим, что по теореме~\ref{comparing-ordinals} один из
фрагментов \textsl{изоморфен} начальному отрезку другого. Пусть
$S$~изоморфен начальному отрезку~$T$ и $h\colon S\hm\to T$\т
их изоморфизм. Лемма утверждает, что изоморфизм~$h$
является тождественным, то есть что $h(x)\hm=x$ при всех~$x\hm\in
S$. Докажем это индукцией по~$x\hm\in S$ (это законно, так как $S$~вполне
упорядочено по определению корректного фрагмента).
Индуктивное предположение гарантирует, что~$h(y)\hm=y$ для всех~$y\hm<x$.
Мы хотим доказать, что~$h(x)\hm=x$. Рассмотрим начальные
отрезки $[0,x)_{S}$ и~$[0,h(x))_{T}$ (с точки зрения порядков~$\le_S$
и~$\le_T$ соответственно).
Они соответствуют друг другу
при изоморфизме~$h$, поэтому по предположению индукции совпадают
как множества. Но по определению корректности
$x\hm=\varphi([0,x))$
и $h(x)\hm=\varphi([0,h(x)))$, так что~$x\hm=h(x)$. Лемма~1 доказана.

Рассмотрим объединение всех корректных фрагментов (как
множеств). На этом объединении естественно определён линейный
порядок: для всяких двух элементов найдётся фрагмент, которому
они оба принадлежат (каждый принадлежит своему, возьмём больший
из фрагментов), так что их можно сравнить. По лемме~1 порядок не
зависит от того, какой фрагмент будет выбран для сравнения.

\textsf{Лемма~2}. Это объединение будет корректным фрагментом.

Чтобы доказать лемму~2, заметим, что на этом объединении
определён линейный порядок. Он будет полным. Для разнообразия
объясним это в терминах убывающих (невозрастающих)
последовательностей. Пусть $x_0\hm\ge x_1\hm\ge \ldots$; возьмём
корректный фрагмент~$F$, которому принадлежит~$x_0$. Из леммы~1
следует, что все~$x_i$ также принадлежат этому фрагменту
(поскольку фрагмент~$F$ будет начальным отрезком в любом большем
фрагменте), а $F$~вполне упорядочен по определению, так что
последовательность стабилизируется. Лемма~2 доказана.

Утверждение леммы~2 можно переформулировать таким образом:
существует наибольший
корректный фрагмент. Осталось доказать, что этот фрагмент
(обозначим его~$S$) включает в себя всё множество~$A$.
Если~$S\hm\ne A$, возьмём элемент~$a\hm=\varphi(S)$, не принадлежащий~$S$,
и добавим его к~$S$, считая, что он больше всех
элементов~$S$. Полученное упорядоченное множество~$S'$
(сумма~$S$ и одноэлементного множества) будет, очевидно,
вполне упорядочено. Кроме того, условие корректности
также выполнено (для~$a$\т по построению, для остальных
элементов\т поскольку оно было выполнено в~$S$).
Таким образом, мы построили больший корректный фрагмент,
что противоречит максимальности~$S$. Это рассуждение
завершает доказательство теоремы~Цермело.
        %
\end{proof}

Как мы уже говорили, из теоремы~Цермело и
теоремы~\ref{comparing-ordinals} о сравнении вполне
упорядоченных множеств
немедленно вытекает такое утверждение:

\begin{theorem}
        \label{zermelo-cardinals-comparing}
Из любых двух множеств одно равномощно подмножеству другого.
        %
\end{theorem}

\begin{problem}
     \label{cantor-bernstein-alternative}
Докажите теорему Кантора\ч Бернштейна (теорема~\ref{cantor-bernstein}), используя теорему Цермело и теорему~\ref{comparing-ordinals}. (Указание. Достаточно доказать, что если $A\subset B\subset C$ и $A$ равномощно $C$, то все три множества равномощны. Можно считать, что $C$ вполне упорядочено и является кардиналом. Задача~\ref{subset-initial} позволяет считать $B$ начальным отрезком $C$, а $A$\т начальным отрезком $B$, и тогда оба этих отрезка должны совпасть со всем $C$.)
     %
\end{problem}

\begin{historyremark}
        %
Понятие вполне упорядоченного множества\index{Вполне упорядоченные
множества}\index{Множества!вполне упорядоченные} ввёл
Кантор\glossary{Кантор@\Кантор}
в работе 1883~года; в его итоговой работе 1895\ч1897~годов приводится
доказательство того, что любые два вполне упорядоченных
множества сравнимы (одно изоморфно начальному отрезку другого).

Утверждения о возможности полного упорядочения любого множества
и о сравнении мощностей (теоремы~\ref{theorem-zermelo}
и~\ref{zermelo-cardinals-comparing}) неоднократно встречаются в
работах Кантора, но никакого внятного доказательства он не
предложил, и оно было дано лишь в~1904 году немецким математиком
Э.\,Цермело\glossary{Цермело@\Цермело}.
        %
\end{historyremark}

\section{Трансфинитная индукция и базис~Гамеля}
     \index{Трансфинитная индукция (рекурсия)}%
     \index{Базис Гамеля}%
     \label{hamel-base}%
     \glossary{Гамель@\Гамель}%

Вполне упорядоченные множества и теорема~Цермело позволяют
продолжать индуктивные построения в трансфинитную область (если
выражаться торжественно). Поясним это на примере из линейной
алгебры.

Всякое линейно независимое множество векторов в конечномерном
пространстве может быть дополнено до базиса. Как это
доказывается? Пусть $S$\т данное нам линейно независимое
множество. Если оно не является базисом, то некоторый
вектор~$x_0$ через него не выражается. Добавим его к~$S$,
получим линейно независимое множество $S\cup\{x_0\}$. Если и оно
не является базисом, то некоторый вектор $x_1$ через него не
выражается, \итд Либо на каком\д то шаге мы получим базис, либо
процесс не оборвётся и мы получим бесконечную последовательность
линейно независимых векторов, что противоречит конечномерности.

Теперь с помощью трансфинитной индукции (точнее, рекурсии) мы
избавимся от требования конечномерности.

Пусть дано произвольное векторное пространство.
Го\-во\-рят, что множество (возможно, бесконечное) векторов
\emph{линейно независимо}\index{Линейно независимое множество}%
\index{Множества!линейно независимые}, если никакая нетривиальная
линейная комбинация конечного числа векторов из этого множества не равна
нулю. (Заметим в скобках, что говорить о бесконечных линейных
комбинациях в принципе можно лишь если в пространстве определена
сходимость, чего мы сейчас не предполагаем.) Линейно независимое
множество векторов называется \emph{базисом~Гамеля}\index{Базис Гамеля}%
\glossary{Гамель@\Гамель} (или просто
\emph{базисом}) данного пространства, если любой вектор представим
в виде конечной линейной комбинации элементов этого множества.

Как и в конечной ситуации, максимальное линейно независимое
множество (которое становится линейно зависимым при добавлении
любого нового элемента) является, очевидно, базисом.

\begin{theorem}
        \label{hamel-base-existence}
Всякое линейно независимое множество векторов
может быть расширено до базиса~Гамеля.
        %
\end{theorem}

\begin{proof}
        %
Пусть $S$\т линейно независимое подмножество векторного
пространства~$V$. Рассмотрим вполне упорядоченное множество~$I$
достаточно большой мощности (большей, чем мощность
пространства~$V$). Определим функцию~$f$ из~$I$ в~$V$ с помощью
трансфинитной рекурсии:
        %
\quot{%
        %
$f(i)={}$элемент пространства~$V$, не выражающийся линейно через
          элементы~$S$ и значения~$f(j)$ при~$j\hm<i$.
        }
\noindent
Заметим, что это рекурсивное правило оставляет~$f(i)$
неопределённым, если такого невыразимого элемента не существует.
(Кроме того, можно отметить, что мы снова используем аксиому
выбора. Более подробно следовало бы сказать так: по аксиоме
выбора существует некоторая функция, которая по каждому
подмножеству пространства~$V$, через которое не всё~$V$
выражается, указывает один из невыразимых элементов. Затем эта
функция используется в рекурсивном определении. Впрочем, аксиома
выбора и так уже использована для доказательства теоремы~Цермело.)

Это определение гарантирует, что $f$~является
инъекцией; более того можно утверждать, что все значения~$f$
вместе с множеством~$S$
образуют линейно независимое множество. В самом деле, пусть
линейная комбинация некоторых значений функции~$f$ и элементов множества~$S$
равна нулю.
Можно считать, что все коэффициенты в этой комбинации отличны от
нуля (отбросив нулевые слагаемые). Входящие в комбинацию значения функции~$f$
имеют вид~$f(i)$ при различных~$i$. Посмотрим на тот из них,
который имеет наибольшее~$i$; по построению он должен быть
линейно независим от остальных\т противоречие.

Поскольку мы предположили, что множество~$I$ имеет большую
мощность, чем~$V$, рекурсивное определение
задаёт функцию не на всём~$I$, а только на некотором начальном
отрезке~$[0,i)$, а в точке~$i$ рекурсивное правило не определено
(теорема~\ref{partial-transfinite-recursion}). Это означает, что
все векторы пространства~$V$ выражаются через элементы множества~$S$
и значения
функции~$f$ на промежутке~$[0,i)$. Кроме того, как мы видели, все эти
векторы независимы. Таким образом, искомый базис найден.
        %
\end{proof}

На самом деле можно обойтись без множества большей мощности,
упорядочив само пространство~$V$. При этом на каждом шаге
рекурсии надо либо добавлять очередной элемент к будущему
базису (если он не выражается через предыдущие), либо оставлять
базис без изменений.

\begin{problem}
        \label{hamel-alternative}
Проведите это рассуждение подробно.
        %
\end{problem}

\problskip
Базис~Гамеля может быть использован для построения разных
экзотических примеров. Вот некоторые из них:

\begin{theorem}\label{pathologic-linear}
        %
Существует (всюду определённая) функция $f\colon \bbR\hm\to\bbR$, для которой
$f(x+y)\hm=f(x)\hm+f(y)$ при всех~$x$ и~$y$, но которая не есть
умножение на константу.
        %
\end{theorem}

\begin{proof}
        %
Рассмотрим~$\bbR$ как векторное пространство над полем~$\bbQ$.
В нём есть базис~Гамеля. Пусть $\alpha$\т один
из векторов базиса. Рассмотрим функцию~$f$, которая с
каждым числом~$x$ (рассматриваемым как вектор в пространстве~$\bbR$
над полем~$\bbQ$) сопоставляет его
$\alpha$\д координату (коэффициент при~$\alpha$ в единственном
выражении~$x$ через векторы базиса). Эта функция линейна
над~$\bbQ$, поэтому $f(x+y)\hm=f(x)\hm+f(y)$ для всех~$x,y\hm\in\bbR$.
Она отлична
от нуля ($f(\alpha)\hm=1$) и
принимает лишь рациональные значения, поэтому не может быть
умножением на константу.
        %
\end{proof}

\begin{problem}
        %
Покажите, что всякая функция, обладающая указанными в
теореме~\ref{pathologic-linear} свойствами, не ограничена ни на
каком отрезке и, более того, её график всюду плотен в~$\mathbb{R}^2$.
        %
\end{problem}

\problskip
\begin{theorem}
        \label{r-plus-r}%
Аддитивные группы~$\bbR$ и~$\bbR\hm\oplus\bbR$
изоморфны.
        %
\end{theorem}

\begin{proof}
        %
Рассмотрим~$\bbR$ как векторное пространство над~$\bbQ$ и
выберем базис в этом пространстве. Очевидно, он бесконечен.
Базис в~$\bbR\hm\oplus\bbR$ может быть
составлен из двух частей, каждая из которых представляет собой
базис в одном из экземпляров~$\bbR$.
Как мы увидим чуть позже
(см.~раздел~\ref{cardinal-operations-revisited}), для любого
бесконечного множества~$B$ удвоенная мощность~$B$ (мощность
объединения двух непересекающихся множеств, равномощных~$B$)
равна мощности~$B$.
Наконец, осталось заметить,
что пространства над одним и тем же
полем с равномощными базисами изоморфны как векторные пространства и
тем более как группы.
        %
\end{proof}

\begin{problem}
        %
Докажите, что любой базис в пространстве~$\mathbb{R}$ над
полем~$\mathbb{Q}$ имеет мощность континуума. (При доказательстве
пригодятся результаты раздела~\ref{cardinal-operations-revisited}.)
        %
\end{problem}

\problskip
Мы видели, что трансфинитная индукция позволяет доказать
существование базиса в любом векторном пространстве. Продолжая
эту линию, можно доказать, что любые два базиса векторного
пространства равномощны. (Таким образом, понятие размерности как
мощности базиса корректно определено и для бесконечномерных
векторных пространств.) Мы вернёмся к этому позже, на
с.~\pageref{dimension-correctness}
(теорема~\ref{dimension-correctness}).

Отметим, что существование базиса~Гамеля можно использовать и
\лк в мирных целях\пк, а не только для построения экзотических
примеров. Известная \лк третья проблема~Гильберта\index{Третья
проблема Гильберта}\glossary{Гильберт@\Гильберт}\пк\ состояла в
доказательстве того, что многогранники равного объёма могут не
быть равносоставлены\index{Равносоставленные многогранники}. (Это значит,
что один из них нельзя
разрезать на меньшие многогранники и сложить из них другой
многогранник.) Для многоугольников на плоскости ситуация иная:
если два многоугольника равновелики (имеют равную площадь),
то они равносоставлены.

\begin{theorem}
        %
Куб нельзя разрезать на многогранные части, из которых можно было бы составить правильный тетраэдр.
        %
\end{theorem}

\begin{proof}
        %
Введём понятие \emph{псевдообъёма}\index{Псевдообъём} многогранника. Как
и объём, псевдообъём будет аддитивен (если многогранник разбит
на части, сумма их псевдообъёмов равна псевдообъёму исходного
многогранника); псевдообъёмы равных многогранников будут равны.
Отсюда следует, что псевдообъёмы равносоставленных
многогранников будут равны. Мы подберём псевдообъём так,
чтобы у куба он равнялся нулю, а у тетраэдра нет\т и
доказательство будет завершено.

Псевдообъём многогранника мы определим как сумму $\sum l_i
\varphi(\alpha_i)$, где сумма берётся по всем рёбрам многогранника,
$l_i$\т длина $i$\д го ребра, $\alpha_i$\т двугранный угол при
этом ребре, а $\varphi$\т некоторая функция. Такое определение
автоматически гарантирует, что равные многогранники имеют равные
псевдообъёмы. Что нужно от функции~$\varphi$, чтобы псевдообъём был
аддитивен? Представим себе, что многогранник разрезается
плоскостью на две части, и плоскость проходит через уже
имеющееся ребро длины~$l$. Тогда двугранный угол~$\alpha$ при этом ребре
разбивается на две части~$\beta$
и~$\gamma$. Поэтому в выражении для псевдообъёма вместо
слагаемого $l\varphi(\alpha)$ появляются
слагаемые~$l\varphi(\beta)\hm+l\varphi(\gamma)$, и $\varphi(\alpha)$~должно
равняться~$\varphi(\beta)+\varphi(\gamma)$.
Кроме того, разрезающая плоскость может образовать новое ребро,
пересекшись с какой\д то гранью. Обозначим длину этого ребра
за~$l'$. Тогда в псевдообъёме появятся слагаемые
$l'\varphi(\alpha)\hm+l'\varphi(\pi\hm-\alpha)$ (два
образовавшихся двугранных угла дополнительны), которые в сумме должны
равняться нулю.

Теперь ясно, какими свойствами должна обладать функция~$\varphi$.
Нужно, чтобы $\varphi(\beta\hm+\gamma)\hm=\varphi(\beta)\hm+\varphi(\gamma)$ и
чтобы~$\varphi(\pi)\hm=0$. Тогда псевдообъём будет и впрямь аддитивен.
Аккуратная проверка требует точного определения понятия
многогранника (что не так и просто), и мы её проводить не будем. Наглядно
аддитивность
кажется очевидной, особенно если учесть, что все разрезы можно
проводить плоскостями (при этом могут получиться более мелкие
части, но это не страшно).

Итак, для завершения рассуждения достаточно построить функцию
$\varphi\colon\bbR\hm\to\bbR$, для которой
        %
\begin{itemize}
        %
\item
$\varphi(\beta\hm+\gamma)\hm=\varphi(\beta)\hm+\varphi(\gamma)$ для всех
       $\beta,\gamma\hm\in\bbR$;
        %
\item
$\varphi(\pi)\hm=0$ (это свойство вместе с предыдущим гарантирует
аддитивность псевдообъёма);
        %
\item
$\varphi(\pi/2)\hm=0$ (псевдообъём куба равен нулю; это свойство,
впрочем, легко следует из двух предыдущих);
        %
\item
$\varphi(\theta)\hm\ne 0$, где $\theta$\т двугранный угол при
ребре правильного тетраэдра.
        %
\end{itemize}

Существенно здесь то, что отношение~$\theta/\pi$ иррационально.
Проверим это. Высоты двух соседних граней, опущенные на общее
ребро, образуют равнобедренный треугольник со сторонами
$\sqrt{3}$, $\sqrt{3}$, $2$; надо доказать, что углы этого
треугольника несоизмеримы с~$\pi$. Удобнее рассмотреть не~$\theta$,
а другой угол треугольника
(два других угла треугольника
равны); обозначим
его~$\beta$. Это угол прямоугольного треугольника со сторонами
$1$, $\sqrt{2}$ и~$\sqrt{3}$, так что $$(\cos \beta \hm+
i\sin\beta)=(1\hm+\sqrt{-2})/\sqrt{3}.$$ Если бы угол~$\theta$ был
соизмерим с~$\pi$, то и $\beta$~был бы соизмерим, поэтому
некоторая степень этого комплексного числа равнялась бы единице.
Можно проверить, однако, что это не так, поскольку кольцо
чисел вида~$m\hm+n\sqrt{-2}$ ($m,n\hm\in \bbZ$) евклидово и
разложение на множители в нём однозначно.

Дальнейшее просто: рассмотрим числа~$\pi$ и~$\theta$. Они
независимы как элементы векторного пространства~$\bbR$ над~$\bbQ$,
дополним их до базиса и рассмотрим $\bbQ$\д линейный
функционал $\varphi\colon\bbR\hm\to\bbQ$, равный коэффициенту при~$\theta$
в разложении по этому базису. Очевидно, все требования
при этом будут выполнены.
        %
\end{proof}

\begin{problem}
        %
Покажите, что некоторое усложнение этого рассуждения позволяет
обойтись без базиса~Гамеля: достаточно определять~$\varphi$
не на всех действительных числах, а только на линейных
комбинациях углов, встречающихся при разрезании куба и
тетраэдра на части.
        %
\end{problem}

\smallskip
В качестве отступления приведём
ещё два странных примера, возникающих благодаря аксиоме
выбора (в них даже трансфинитная индукция не используется).\index{Аксиома!выбора}

\textsf{Разноцветные шляпы}. Представим себе сначала шеренгу из
ста человек в чёрных и белых шляпах. Каждый видит цвета шляп
стоящих перед ним, но не видит своей шляпы и шляп позади
стоящих, и должен отгадать цвет своей шляпы. Могут ли участники
игры договориться заранее, как они должны действовать, чтобы с
гарантией большинство догадок было правильными? (Игроки не
произносят вслух свои догадки, так что другие игроки не могут их
учитывать.)

Легко понять, что стоящая перед игроками задача неразрешима.
Стоящий первым в шеренге не видит никого, так что его догадка ни
от чего не зависит и заранее предопределена. Наденем ему шляпу
другого цвета и сделаем его догадку ошибочной. Цвет шляпы
первого определяет догадку второго (у которого нет иной
информации), и ему тоже можно надеть шляпу другого цвета. После
этого определится догадка третьего, и так далее. Таким образом,
для любой стратегии действий игроков есть вариант, в котором все
ошибутся.

Ситуация удивительным образом меняется, если игроки стоят
в бесконечной шеренге, у которой есть последний, но нет первого,
и каждый из игроков видит шляпы впереди стоящих.
Другими словами, игроки стоят в точках $0,1,2,\ldots$ числовой
прямой, глядя вправо, и игрок в точке $i$ знает свою координату
и цвета шляп всех следующих игроков ($i+1,i+2,\ldots$). Его
стратегия, таким образом, является функцией, аргумент которой\т
последовательность цветов видимых им шляп, а значением\т догадка
о цвете собственной шляпы.

Как ни странно, существуют стратегии игроков, при которых с
гарантией все догадки, кроме конечного числа, будут правильными.
Чтобы доказать существование таких стратегий, будем называть \лк
раскраской\пк\ последовательность цветов шляп всех игроков. Две
раскраски будем называть эквивалентными, если они различаются
лишь в конечном числе точек (для конечного числа игроков).
Выберем в каждом классе эквивалентности по одной \лк
канонической\пк\ раскраске. Каждый игрок должен действовать так:
видя всех игроков перед собой, он может определить класс
эквивалентности раскраски (так как изменение конечного числа
остальных игроков, включая его самог\'о, не меняет этого
класса). Затем в качестве догадки он указывает цвет своей шляпы
в каноническом представителе этого класса эквивалентности.

Какова бы ни была реальная раскраска, она отличается от
эквивалентной ей канонической раскраски лишь в конечном числе
мест (по определению эквивалентности). Поэтому все
игроки, стоящие дальше этих мест, угадают правильно\т у них
самих тот же цвет, и видят они то же самое, что в канонической
раскраске.

\begin{problem}
    %
Покажите, что игроки могут добиться того же результата
(гарантировать конечное число неверных догадок) и в более
сложной ситуации, когда они не знают своего положения. (Полезно
отдельно рассмотреть классы, содержащие периодические
раскраски.)
    %
\end{problem}


\textsf{Неизмеримое множество}. Будем считать две точки
окружности эквивалентными, если дуга между ними составляет
рациональную часть окружности (=\,измеряется рациональным числом
градусов). Это действительно отношение эквивалентности, так как
сумма двух рациональных чисел рациональна (транзитивность).
Выберем в каждом классе эквивалентности по одной точке;\index{Аксиома!выбора}
полученное множество \лк представителей\пк\ обозначим $M$.

Если повернуть множество $M$ на рациональный (в градусах, то
есть соизмеримый с окружностью) угол $\alpha$, то повёрнутое
множество $M_\alpha$ не будет пересекаться с исходным:
пересечение означало бы, что есть два представителя,
отличающихся на $\alpha$, то есть лежащих
в одном и том же классе.

С другой стороны, в объединении множества $M_\alpha$ (при всех
рациональных углах поворота $\alpha$) дают всю окружность
(каждая точка имеет представителя в своём классе, и поворот $M$
на соответствующий угол покроет её).

Получается, что окружность разбита на счётное число конгруэнтных
(совмещающихся поворотом) множеств. Какую же долю окружности (по
мере) составляет каждое множество? Поскольку множества
конгруэнтны, то эти доли равны; поскольку они все помещаются в
окружности без пересечений, то доля не может быть
положительной, то есть равна нулю.

Как ни странно, в объединении счётное число таких \лк нулевых\пк\
множеств составляет всю окружность. (Говоря более формально, не
существует ненулевой счётно\д аддитивной меры, определённой на
всех подмножествах окружности и инвариантной относительно
поворотов.)

Ещё удивительнее пространственный вариант этого
примера: можно разбить шар на конечное число непересекающихся
множеств, из которых (сдвинув каждую часть в пространстве) можно
получить два шара того же размера. Этот пример называют \лк
парадоксом Банаха\ч
Тарского\пк.\glossary{Банах@\Банах}\glossary{Тарский@\Тарский}\index{Банаха\ч
Тарского парадокс}\index{Парадокс!Банаха\ч Тарского} Но это
построение сложнее и требует знакомства с алгеброй (в одном из
вариантов используется свободная подгруппа с двумя образующими в
группе $SO(3)$).

\section{Лемма~Цорна и её применения}\index{Лемма Цорна}
        \label{zorn-applications}%
        \glossary{Цорн@\Цорн}%

В современных учебниках редко встречается трансфинитная индукция
как таковая: она заменяется ссылкой на так называемую лемму~Цорна.
Сейчас мы покажем, как это делается, на примере теоремы о
существовании базиса в линейном пространстве.

\begin{theorem}[лемма Цорна]\glossary{Цорн@\Цорн}
        \label{zorn-lemma}%
        \index{Лемма Цорна}%
Пусть $Z$\т частично упорядоченное множество, в
котором всякая цепь имеет верхнюю границу. Тогда в этом
множестве есть максимальный элемент, и, более того, для любого
элемента~$a\hm\in Z$ существует элемент~$b\hm\ge a$, являющийся
максимальным в~$Z$. (\emph{Цепь}\index{Цепь}\т это подмножество,
любые два элемента
которого сравнимы. Верхняя граница цепи\т элемент, больший или
равный любого элемента цепи.)
        %
\end{theorem}

\begin{proof}
        %
Прежде всего отметим, что $Z$~лишь частично упорядочено, поэтому
надо различать максимальные и наибольшие элементы. По этой же
причине мы вынуждены употреблять грамматически некорректную
конструкцию \лк больший или равный любого (любому?)\пк, поскольку
сказать \лк не меньше любого\пк\ (стандартный выход из
положения) означало бы изменить смысл.

Доказательство повторяет рассуждения при построении базиса,
но в более общей ситуации (теперь у нас не линейно независимые
семейства, а произвольные элементы~$Z$).

Пусть дан произвольный элемент~$a$. Предположим, что не существует
максимального элемента, большего или равного~$a$. Это значит,
что для любого~$b\hm\ge a$ найдётся~$c\hm>b$. Тогда $c\hm>a$ и потому
найдётся~$d\hm>c$~\итд Продолжая этот процесс достаточно долго, мы исчерпаем
все элементы~$Z$ и придём к противоречию.

Проведём рассуждение аккуратно (пока что мы даже не использовали
условие леммы, касающееся цепей).
Возьмём вполне упорядоченное множество~$I$ достаточно
большой мощности (большей, чем мощность~$Z$).
Построим строго возрастающую функцию $f\colon I\hm\to Z$ по
трансфинитной рекурсии. Её значение на минимальном элементе~$I$
будет равно~$a$. Предположим, что мы уже знаем все её значения
на всех элементах, меньших некоторого~$i$. В силу монотонности
эти значения попарно сравнимы. Поэтому существует их верхняя
граница~$s$, которая, в частности, больше или равна~$a$. Возьмём
какой\д то элемент~$t\hm>s$ и положим~$f(i)\hm=t$; по построению
монотонность сохранится. Тем самым $I$~равномощно части~$Z$, что
противоречит его выбору.

В этом рассуждении, формально говоря, есть пробел: мы
одновременно определяем функцию по трансфинитной рекурсии и
доказываем её монотонность с помощью трансфинитной индукции.
Наше рекурсивное определение имеет смысл, лишь если уже
построенная часть функции монотонна. Формально говоря, надо
воспользоваться теоремой~\ref{partial-transfinite-recursion},
считая, что следующее значение не определено, если уже
построенный участок не монотонен, и получить функцию,
определённую на всём~$I$ или на начальном отрезке. Если она
определена на некотором начальном отрезке, то она монотонна на
нём по построению, поэтому следующее значение тоже определено\т
противоречие.
        %
\end{proof}

Как и при построении базиса~Гамеля\index{Базис Гамеля}
(задача~\ref{hamel-alternative},
с.~\pageref{hamel-alternative}), можно обойтись без множества большей
мощности. Вполне упорядочим множество~$Z$ с помощью теоремы~Цермело.
Этот порядок никак не связан с исходным порядком на~$Z$; мы будем обозначать
его символом~$\prec$. Построим с помощью трансфинитной рекурсии
функцию $f\colon Z \hm\to Z$ с такими свойствами:
        %
(1)~$f(z) \hm\ge a$ для любого~$z\hm\in Z$;
        %
(2)~$f$~монотонна в следующем смысле: если $x\hm\prec y$, то
$f(x)\hm\le f(y)$;
        %
(3)~$f(z)$ не может быть строго меньше~$z$ (в смысле исходного порядка~$\le$)
ни при каком~$z$.

Делается это так. Значение~$f(z_0)$ для $\prec$\д наименьшего
элемента~$z_0$ мы положим равным либо~$a$, либо~$z_0$
(последнее\т если~$z_0 \hm> a$). Значение~$f(z)$ для остальных~$z$
есть либо верхняя граница значений~$f(z')$ при~$z'\hm\prec z$
(по предположению
индукции множество таких значений линейно
упорядочено и потому имеет некоторую верхнюю
границу~$\alpha$), либо само~$z$ (последнее\т если~$z\hm>\alpha$).

В силу монотонности множество значений функции~$f$ линейно
упорядочено и имеет верхнюю границу. Эта граница (обозначим
её~$\beta$) больше или равна~$a$ (которое есть~$f(z_0)$) и
является искомым максимальным элементом: если~$\beta \hm< z$ для
некоторого~$z$, то~$f(z)\hm\le\beta\hm<z$, что противоречит
свойству~(3).

\begin{problem}
        %
Проведите это рассуждение подробно.
        %
\end{problem}

\problskip
Теперь повторим доказательство теоремы о базисе, используя лемму~Цорна.
Пусть~$V$\т произвольное векторное пространство.
Рассмотрим частично упорядоченное множество~$Z$, состоящее из
линейно независимых подмножеств пространства~$V$. Порядок
на~$Z$ задаётся отношением~\лк быть подмножеством\пк.

Проверим, что условия леммы выполнены. Пусть имеется некоторая
цепь, то есть семейство линейно независимых множеств, причём
любые два множества этого семейства сравнимы. Объединим все эти
множества и покажем, что полученное множество будет линейно
независимым (тем самым оно будет верхней границей элементов
цепи). В самом деле, нетривиальная линейная комбинация включает
в себя какое\д то конечное число векторов, каждый из своего
множества. Этих множеств конечное число, и потому среди них есть
наибольшее по включению (в конечном линейно упорядоченном
множестве есть наибольший элемент). Это наибольшее множество
содержит все векторы нетривиальной линейной комбинации, и
линейно независимо по предположению, так что наша нетривиальная
линейная комбинация отлична от нуля.

Таким образом, можно применить лемму Цорна и заключить, что
любое линейно независимое множество векторов содержится в
максимальном линейно независимом множестве векторов. К нему
уже нельзя добавить ни одного вектора, не создав линейной
зависимости, и оно является искомым
базисом.

Аналогичным образом можно доказать существование ортогонального
базиса в
гильбертовом пространстве (там определение базиса другое: разрешаются
бесконечные линейные комбинации, понимаемые как суммы рядов) или
существование базиса трансцендентности (максимальная
алгебраически независимая система элементов в расширении полей).

Мы приведём другой пример применения леммы~Цорна, где
фигурируют уже известные нам понятия.

\begin{theorem}
        \label{partial-to-linear}%
Всякий частичный порядок\index{Частичный порядок} может быть продолжен до
        \index{Отношение!частичного порядка}%
        \index{Отношение!линейного порядка}%
линейного\index{Линейный порядок}.
        %
\end{theorem}

\begin{proof}
        %
Пусть $(X,\le)$\т частично упорядоченное множество. Теорема
утверждает, что существует отношение порядка~$\le'$ на~$X$,
продолжающее исходное (это значит, что~$x\hm\le y\Rightarrow x \le'
y$) и являющееся отношением линейного порядка. (Кстати,
отметим, что слово \лк линейного\пк\ в формулировке теоремы
нельзя заменить на слово \лк полного\пк\т например, если исходный
порядок линейный, но не полный.)

Готовясь к применению леммы~Цорна, рассмотрим частично
упорядоченное множество~$Z$, элементами которого будут частичные
порядки на~$X$ (то есть подмножества множества~$X\hm\times X$, обладающие
свойствами рефлексивности, транзитивности и антисимметричности),
упорядоченные по включению: $\le_1$~считается меньшим или
равным~$\le_2$, если $\le_2$~продолжает $\le_1$ (из $x\le_1 y$
следует~$x\le_2 y$).

Легко проверить, что условие леммы~Цорна выполнено: если
у нас есть семейство частичных порядков, линейно упорядоченное
по включению, то объединение этих порядков является частичным
порядком, и этот порядок будет верхней границей семейства.
(Проверим, например, что объединение обладает свойством
транзитивности. Пусть $x\le_1 y$ в одном из порядков семейства $(\le_1)$,
а $y\le_2 z$ в другом; один из порядков (например,~$\le_1$) продолжает
другой, тогда $x\le_1 y \le_1 z$ и потому $x\hm\le z$ в объединении.
Рефлексивность и антисимметричность проверяются столь же просто.)

Следовательно, по лемме~Цорна на множестве~$X$ существует
максимальный частичный порядок, продолжающий
исходный. Обозначим
его как $\le$ (путаницы с исходным порядком не возникнет,
так как исходный нам больше не нужен). Нам надо показать, что он
будет линейным. Пусть
$x,y\hm\in X$\т два несравнимых элемента. Расширим порядок до
нового порядка~$\le'$, при котором $x\le' y$. Этот новый порядок
определяется так:
$a\le'
b$, если (1)~$a\le b$ или (2)~$a\le x$ и~$y\hm\le b$.
Несложно проверить, что $\le'$~будет
частичным порядком. Рефлексивность очевидна. Транзитивность:
если $a\le'b$ и $b\le' c$, то есть четыре возможности. Если в
обоих случаях имеет место случай~(1), то $a\hm\le b \hm\le c$ и всё
очевидно. Если $a\le' b$ в силу~(1), а $b\le c$ в силу~(2), то
$a\hm\le b\hm\le x$ и $y\hm\le c$, так что $a \le' c$ в силу~(2).
Аналогично рассматривается и симметричный случай. Наконец,
двукратная ссылка на~(2) невозможна, так как тогда $(a\hm\le x)$,
$(y\hm\le b)$, $(b\hm\le x)$ и~$(y\hm\le c)$, и получается, что $y\hm\le b\hm\le x$,
а мы предполагали, что $x$ и~$y$ не сравнимы. Антисимметричность
доказывается аналогично. Таким образом, отношение~$\le'$ будет
частичным порядком, строго содержащим~$\le$, что противоречит
максимальности.
        %
\end{proof}

\begin{problem}
        %
Покажите, что любое бинарное отношение без циклов (цикл
образуется, если $xRx$, или $xRyRx$, или $xRyRzRx$ \итд) может
быть продолжено до линейного порядка. (Для конечных множеств
поиск такого продолжения обычно называют \лк топологической
сортировкой\index{Топологическая сортировка}\пк.)
        %
\end{problem}

\begin{problem}
        %
Множество на плоскости называется \emph{выпуклым}, если вместе с любыми
двумя точками оно содержит соединяющий их отрезок. Покажите, что
любые два непересекающихся выпуклых множества можно разделить прямой
(каждое множество лежит по одну сторону от прямой, возможно, пересекаясь
с ней). (Указание. Используя лемму Цорна, можно расширить исходные
непересекающиеся множества~$A$ и~$B$ до взаимно дополнительных
выпуклых множеств~$A'$ и $B'$. Затем можно убедиться, что
граница между~$A'$ и~$B'$ представляет собой прямую.)
        %
\end{problem}

\begin{problem}
        %
Покажите, что все подмножества натурального ряда можно разбить
на \лк большие\пк\ и \лк малые\пк\ таким образом, чтобы
выполнялись следующие свойства: (1)~множество большое тогда и
только тогда, когда его дополнение мало; (2)~любое подмножество
малого множества мало, а любое надмножество большого множества\т
большое; (3)~объединение двух малых множеств мало, а пересечение
двух больших множеств\т большое; (4)~все конечные множества\т
малые, а все множества с конечными дополнениями\т большие.
(Вначале будем требовать условий (1)\ч (4), но не настаивать на
том, что любое множество должно быть большим или малым.
Например, можно считать малыми конечные множества, а большими\т
их дополнения. Затем можно добавлять множества, определяя их в
большие или малые, применив трансфинитную индукцию или лемму
Цорна.)
       %
\end{problem}

Построенное в этой задаче деление множеств на большие и малые
(точнее, класс больших множеств) называют \emph{неглавным
ультрафильтром};\index{ультрафильтр} слово \лк ультра\пк\ здесь
соответствует требованию того, что всякое множество большое или
малое; слово \лк неглавный\пк\ обозначает требование~(4).

\begin{problem}
        %
(Продолжение.) Два агрессора играют в такую игру: сначала первый
захватывает начальный отрезок натурального ряда произвольного
размера (от $0$ до какого\д то $k$), затем второй захватывает
примыкающий отрезок (от $k+1$ до какого\д то $l$, затем первый
захватывает следующий (от $l+1$ до какого\д то $m$), \итд Игра
бесконечна; победителем в бесконечной партии считается тот
игрок, который захватил большое множество. Покажите, что в этой
игре ни один из игроков не имеет выигрышной стратегии.
(Выигрышную стратегию для одного мог бы использовать и другой,
поскольку конечный начальный участок не влияет на исход игры.)
        %
\end{problem}


\section{Свойства операций над
мощностями}\label{cardinal-operations-revisited}

Теперь мы можем доказать несколько утверждений о мощностях.

\begin{theorem}
        %
Если $A$~бесконечно, то множество~$A\hm\times\bbN$~равномощно~$A$.
        %
\end{theorem}

\begin{proof}
        %
Вполне упорядочим множество~$A$. Мы уже знаем
(см.~с.~\pageref{limit-natural}), что всякий элемент множества~$A$
однозначно представляется в виде~$z\hm+n$, где $z$\т предельный
элемент (не имеющий непосредственно предыдущего), а $n$\т
натуральное число. Это означает, что $A$~равномощно~$B\hm\times\bbN$,
где $B$\т множество предельных элементов. (Тут есть небольшая
трудность\т последняя группа элементов
конечна, если в множестве есть наибольший элемент. Но мы уже знаем,
что добавление конечного или счётного множества не меняет
мощности, так что этим можно пренебречь.)

Теперь утверждение теоремы
очевидно: $A\hm\times\bbN$~равномощно
$(B\hm\times\bbN)\hm\times\bbN$, то есть
$B\hm\times(\bbN\hm\times\bbN)$ и тем самым
$B\hm\times\bbN$ (произведение счётных множеств
счётно), то есть~$A$.
        %
\end{proof}

По теореме Кантора\ч
Бернштейна\index{Теорема!Кантора\Ч Шрёдера\Ч Бернштейна}
отсюда следует, что
промежуточные мощности (в частности,~$|A|\hm+|A|$, а также любое
произведение~$A$ и конечного
множества) совпадают с~$|A|$. Ещё одно следствие полезно
выделить:

\begin{theorem}
        \label{adding-cardinals}
Сумма\index{Сумма мощностей}\index{Мощности!сумма} двух бесконечных
мощностей равна их максимуму.
        %
\end{theorem}

\begin{proof}
        %
Прежде всего напомним, что любые две мощности сравнимы
(теорема~\ref{zermelo-cardinals-comparing},
с.~\pageref{zermelo-cardinals-comparing}). Пусть, скажем,~$|A|\hm\le |B|$.
Тогда $|B|\hm\le |A|+|B| \hm\le |B|+|B| \hm\le
|B|\hm\times\aleph_0 \hm= |B|$ (последнее неравенство\т утверждение
предыдущей теоремы). Остаётся воспользоваться теоремой Кантора\ч
Бернштейна и заключить, что~$|B|\hm=|A\hm+B|$.
        %
\end{proof}

Теперь можно  доказать более сильное утверждение.

\begin{theorem}
        \label{cardinal-squared}%
Если $A$~бесконечно, то $A\hm\times A$ равномощно~$A$.
        %
\end{theorem}

\begin{proof}
        %
Заметим, что для счётного множества (как, впрочем, и для
континуума\т но это сейчас не важно) мы это уже знаем.
Поэтому в~$A$ есть подмножество, равномощное своему
квадрату.

Рассмотрим семейство всех таких подмножеств вместе с
соответствующими биекциями. Элементами этого семейства будут
пары~$\langle B, f\rangle$, где $B$\т подмножество~$A$, а
$f\colon B\hm\to B\hm\times B$\т взаимно однозначное соответствие. Введём на
этом семействе частичный порядок: $\langle B_1, f_1\rangle\hm\le
\langle B_2,f_2\rangle$, если~$B_1 \hm\subset B_2$ и ограничение
отображения~$f_2$ на~$B_1$ совпадает с~$f_1$ (рис.~\ref{card-1}).

\begin{figure}[ht]
        $$
\unitlength=0.5mm
\begin{picture}(50,55)
\put(0,0){\line(1,0){50}}
\put(0,50){\line(1,0){50}}
\put(0,0){\line(0,1){50}}
\put(50,0){\line(0,1){50}}
        %
\put(20,0){\line(0,1){20}}
\put(0,20){\line(1,0){20}}
        %
\put(10,-7){\hbox to 0pt{\hss $B_1$\hss}}
\put(25,53){\hbox to 0pt{\hss $B_2$\hss}}
\put(-2,7){\hbox to 0pt{\hss $B_1$}}
\put(52,23){\hbox to 0pt{$B_2$\hss}}
        %
\end{picture}
        $$
\caption{
Отображение~$f_1$\т взаимно однозначное соответствие
между малым квадратом и его стороной; $f_2$~добавляет к нему
взаимно однозначное соответствие между $B_2\hm\setminus B_1$ и
\лк уголком\пк\ $(B_2\hm\times B_2)\setminus(B_1\hm\times B_1)$.
        }
\label{card-1}
\end{figure}

Теперь применим лемму~Цорна. Для этого нужно убедиться, что любое
линейно упорядоченное (в смысле описанного порядка) множество
пар указанного вида имеет верхнюю границу. В самом деле,
объединим все первые компоненты
этих пар; пусть $B$\т их
объединение. Как обычно, согласованность отображений
(гарантируемая определением порядка) позволяет соединить
отображения в одно. Это отображение (назовём его~$f$) отображает~$B$
в~$B\hm\times B$. Оно будет инъекцией: значения~$f(b')$ и~$f(b'')$
при различных $b'$ и~$b''$ различны (возьмём большее
из множеств, которым принадлежат $b'$ и~$b''$; на нём $f$~является
инъекцией по предположению). С другой стороны, $f$~является
сюръекцией: для любой пары $\langle b',b''\rangle\hm\in B\hm\times B$
возьмём множества, из которых произошли $b'$ и $b''$, выберем
из них большее и вспомним, что мы имели взаимно однозначное
соответствие между ним и его квадратом.

По лемме~Цорна в нашем частично упорядоченном множестве
существует максимальный элемент. Пусть этот элемент
есть~$\langle B,f\rangle$. Мы знаем, что $f$ есть взаимно однозначное
соответствие между~$B$ и~$B\hm\times B$ и потому~$|B|\hm=|B|\hm\times|B|$.
Теперь есть две возможности. Если $B$~равномощно~$A$,
то $B\hm\times B$ равномощно~$A\hm\times A$ и всё
доказано. Осталось рассмотреть случай, когда $B$~не равномощно~$A$,
то есть имеет меньшую мощность (большей оно иметь не может,
будучи подмножеством). Пусть
$C$\т оставшаяся часть~$A$, то
есть~$A\hm\setminus B$. Тогда $|A|\hm=|B|\hm+|C|\hm=\max(|B|,|C|)$,
следовательно, $C$~равномощно~$A$ и больше~$B$ по мощности.
Возьмём в~$C$ часть~$C'$, равномощную~$B$, и положим~$B'\hm=B\hm+C'$
(рис.~\ref{card-2}).
        %
\begin{figure}[ht]
        $$
\unitlength=0.5mm
\begin{picture}(50,55)
\put(0,0){\line(1,0){50}}
\put(0,50){\line(1,0){50}}
\put(0,0){\line(0,1){50}}
\put(50,0){\line(0,1){50}}
        %
\put(20,0){\line(0,1){40}}
\put(40,0){\line(0,1){40}}
\put(0,20){\line(1,0){40}}
\put(0,40){\line(1,0){40}}
        %
\put(10,-7){\hbox to 0pt{\hss $B$\hss}}
\put(30,-7){\hbox to 0pt{\hss $C'$\hss}}
\put(25,53){\hbox to 0pt{\hss $A$\hss}}
\put(-2,7){\hbox to 0pt{\hss $B$}}
\put(-2,27){\hbox to 0pt{\hss $C'$}}
\put(52,23){\hbox to 0pt{$A$\hss}}
        %
\end{picture}
        $$
\caption{Продолжение соответствия с $B$ на $B'=B+C'$.}
\label{card-2}
\end{figure}
        %
Обе части множества~$B'$ равномощны~$B$. Поэтому $B'\hm\times B'$~разбивается
на $4$~части, каждая из которых равномощна~$B\hm\times B$, и, следовательно,
равномощна~$B$ (напомним, что у нас есть
взаимно однозначное соответствие~$f$ между~$B$ и~$B\hm\times B$).
Соответствие~$f$ можно продолжить до соответствия~$f'$ между~$B'$
и~$B'\hm\times B'$, дополнив его соответствием между~$C'$
и $(B'\hm\times B')\hm\setminus(B\hm\times B)$ (эта разность состоит из трёх
множеств, равномощных~$B$, так что равномощна~$B$).
В итоге мы получаем б\'ольшую пару~$\langle B',f'\rangle$,
что противоречит утверждению леммы~Цорна о максимальности.
Таким образом, этот случай невозможен.
        %
\end{proof}

Выведем теперь некоторые следствия из доказанного утверждения.

\begin{theorem}
        \label{sequences-cardinality}%
(\textsf{а})~%
Произведение\index{Произведение мощностей}\index{Мощности!произведение} двух
бесконечных мощностей равно большей из них.
        %
(\textsf{б})~%
Если множество $A$ бесконечно, то множество $A^n$ всех
последовательностей длины $n\hm>0$, составленных из элементов~$A$,
равномощно~$A$.
        %
(\textsf{в})~%
Если множество~$A$ бесконечно, то множество всех конечных
последовательностей, составленных из элементов~$A$, равномощно~$A$.
        %
\end{theorem}

\begin{proof}
        %
Первое утверждение доказывается просто: если $|A|\hm\le |B|$,
то $|B|\hm\le |A|\hm\times |B| \hm\le |B|\hm\times |B| \hm= |B|$.

Второе утверждение легко доказывается индукцией
по~$n$: если
$|A^n|\hm=|A|$, то $|A^{n+1}|\hm=|A^n|\hm\times |A|\hm=|A|\hm\times |A|\hm=|A|$.

Третье тоже просто: множество конечных последовательностей
есть $1\hm+A\hm+A^2\hm+A^3\hm+\ldots$; каждая из
частей (кроме первой, которой можно пренебречь) равномощна~$A$
(по доказанному), и потому всё вместе
есть $|A|\hm\times\aleph_0\hm=|A|$.
        %
\end{proof}

Заметим, что из последнего утверждения теоремы вытекает, что
семейство всех конечных подмножеств бесконечного множества~$A$
имеет ту же мощность, что и~$A$ (подмножеств не больше, чем
конечных последовательностей и не меньше, чем одноэлементных
подмножеств).

\begin{problem}
        %
Пусть $A$ бесконечно. Докажите, что
$|A^A|=|2^A|$.
        %
\end{problem}

\begin{problem}
        %
Рассмотрим мощность
$\alpha\hm=\aleph_0\hm+2^{\aleph_0}\hm+2^{(2^{\aleph_0})}\hm+\ldots$
(счётная сумма). Покажите, что $\alpha$\т минимальная мощность,
которая больше мощностей множеств $\bbN$, $P(\bbN)$, $P(P(\bbN))$, $\dots$
Покажите, что $\alpha^{\aleph_0}\hm=2^{\alpha}\hm>\alpha$.
        %
\end{problem}

Теперь мы можем доказать упоминавшееся ранее утверждение
о равномощности базисов\index{Равномощность базисов}.
        %
\begin{theorem}
        \label{dimension-correctness}%
Любые два базиса в бесконечномерном векторном пространстве
имеют одинаковую мощность.
        %
\end{theorem}

\begin{proof}
        %
Пусть даны два базиса\т первый и второй. Для каждого вектора
из первого базиса фиксируем какой\Д либо\- способ выразить его
через векторы второго базиса. В этом выражении участвует
конечное множество векторов второго базиса. Таким образом,
есть некоторая функция, которая каждому вектору первого
базиса ставит в соответствие некоторое конечное множество
векторов второго. Как мы только что видели, возможных значений этой
функции столько же, сколько элементов во втором базисе.
Кроме того, прообраз каждого значения состоит из векторов
первого базиса, выражающихся через данный (конечный) набор векторов
второго, и потому конечен. Выходит, что первый базис
разбит на группы, каждая группа конечна, а всего групп не
больше, чем векторов во втором базисе. Поэтому
мощность первого базиса не превосходит мощности второго,
умноженной на~$\aleph_0$ (от чего, как мы знаем, мощность
бесконечного множества не меняется). Осталось провести симметричное
рассуждение и сослаться на теорему Кантора\ч Бернштейна.
        %
\end{proof}
