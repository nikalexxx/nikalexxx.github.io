\chapter{Упорядоченные множества}

\section[Эквивалентность и порядок]%
{Отношения эквивалентности и порядка}\index{Отношение!эквивалентности}%
\index{Отношение!порядка}
        \label{equivalence-order}

Напомним, что бинарным отношением на мно\-жест\-ве~$X$ называется
подмножество $R\hm\subset X\hm\times X$; вместо $\langle
x_1,x_2\rangle \hm\in R$ часто пишут~$x_1 R x_2$.

Бинарное отношение~$R$ на множестве~$X$
называется \emph{отношением эквивалентности}\index{Отношение!эквивалентности},
если выполнены следующие свойства:
        %
\begin{itemize}
        %
\item (рефлексивность)\index{Рефлексивность} $xRx$ для всех~$x\hm\in X$;
        %
\item (симметричность)\index{Симметричность} $xRy \hm\Rightarrow yRx$
для всех~$x,y\hm\in X$;
        %
\item (транзитивность)\index{Транзитивность} $xRy \text{ и } yRz \hm\Rightarrow
        xRz$ для всех~$x,y,z\hm \in X$.
\end{itemize}
         %
Имеет место следующее очевидное, но важное утверждение:

\begin{theorem}
        \label{equivalence-classes}%
(\textsf{а})
        %
Если множество $X$ разбито в объединение непересекающихся
подмножеств, то отношение \лк лежать в одном подмножестве\пк\
является отношением эквивалентности.

(\textsf{б})
        %
Всякое отношение эквивалентности получается описанным способом
из некоторого разбиения.
        %
\end{theorem}

\begin{proof}
        %
Первое утверждение совсем очевидно; мы приведём доказательство
второго, чтобы было видно, где используются все пункты
определения эквивалентности. Итак, пусть $R$\т отношение
эквивалентности. Для каждого элемента~$x\hm\in X$ рассмотрим его
\emph{класс эквивалентности}\index{Класс эквивалентности}\т множество
всех~$y\in X$, для которых верно~$xRy$.

Докажем, что для двух различных $x_1$, $x_2$ такие множества либо
не пересекаются, либо совпадают. Пусть они пересекаются, то есть
имеют общий элемент~$z$. Тогда $x_1 R z$ и~$x_2 R z$, откуда
$z R x_2$ (симметричность) и $x_1 R x_2$ (транзитивность), а также
$x_2 R x_1$ (симметричность). Поэтому для любого~$z$
из~$x_1 R z$ следует~$x_2 R z$ (транзитивность) и наоборот.

Осталось заметить, что в силу рефлексивности каждый элемент~$x$
принадлежит задаваемому им классу, то есть действительно
всё множество~$X$ разбито на непересекающиеся классы.
        %
\end{proof}

\begin{problem}
        %
Покажите, что требования симметричности и транзитивности можно
заменить одним: $\text{$xRz$ и $yRz$} \Rightarrow xRy$
(при сохранении требования рефлексивности).
        %
\end{problem}

\begin{problem}
        %
Сколько различных отношений эквивалентности существует на множестве
$\{1,2,3,4,5\}$?
        %
\end{problem}

\begin{problem}
        %
На множестве~$M$ заданы два отношения эквивалентности,
обозначаемые $\sim_1$ и $\sim_2$, имеющие $n_1$ и $n_2$ классов
эквивалентности соответственно.
Будет ли их пересечение
$x\sim y \hm\Leftrightarrow [(x\sim_1 y)\text{ и }(x\sim_2 y)]$
отношением эквивалентности? Сколько у него может быть классов?
Что можно сказать про объединение отношений?
        %
\end{problem}

\begin{problem}
        %
(Теорема Рамсея\index{Теорема!Рамсея}\glossary{Рамсей@\Рамсей})
Множество всех $k$\д элементных подмножеств
бесконечного множества~$A$ разбито на $l$~классов ($k$, $l$\т
натуральные числа). Докажите, что найдётся бесконечное множество
$B\hm\subset A$, все $k$\д элементные подмножества которого
принадлежат одному классу.

(При~$k\hm=1$ это очевидно: если бесконечное множество разбито
на конечное число классов, то один из классов бесконечен. При~$k=2$
и~$l=2$ утверждение можно сформулировать так:
из бесконечного множества людей можно выбрать либо
бесконечно много попарно знакомых, либо бесконечно много
попарно незнакомых. Конечный вариант этого утверждения\т о том,
что среди любых шести людей есть либо три попарно знакомых,
либо три попарно незнакомых,\т известная задача для школьников.)
        %
\end{problem}

Множество классов эквивалентности называют \emph{фактор\д
множеством}\index{Фактор\д множество}
множества~$X$ по отношению эквивалентности~$R$.
(Если отношение согласовано с дополнительными структурами
на~$X$, получаются фактор\д группы, фактор\д кольца \итд)

Отношения эквивалентности нам не раз ещё встретятся, но сейчас
наша основная тема\т отношения порядка.

Бинарное отношение~$\le$ на множестве~$X$ называется \emph{отношением
частичного порядка},
    \index{Отношение!частичного порядка}%
    \index{Частичный порядок}%
если выполнены такие свойства:
        %
\begin{itemize}
        %
\item (рефлексивность)\index{Рефлексивность} $x\le x$ для всех~$x\hm\in X$;
        %
\item (антисимметричность)\index{Антисимметричность}
         $x\le y \text{ и } y\le x \hm\Rightarrow x\hm=y$\\
         для всех $x,y\hm\in X$;
       %
\item (транзитивность)\index{Транзитивность}
         $x\le y\text{ и }y\le z \hm\Rightarrow
        x\hm\le z$ для всех $x,y,z\hm \in X$.
\end{itemize}
        %
(Следуя традиции, мы используем символ~$\le$ (а не букву) как знак
отношения порядка.) Множество с заданным на нём отношением частичного
порядка называют \emph{частично упорядоченным}.%
        \index{Множества!частично упорядоченные}%

Говорят, что два элемента $x,y$ частично упорядоченного
множества \emph{сравнимы}\index{Сравнимость},
если $x\le y$ или $y\le x$. Заметим,
что определение частичного порядка не требует, чтобы любые два
элемента множества были сравнимы. Добавив это требование, мы получим
определение \emph{линейного порядка} (\emph{линейно упорядоченного
множества}).
        \index{Линейный порядок}%
        \index{Отношение!линейного порядка}%
        \index{Множества!линейно упорядоченные}%

Приведём несколько примеров частичных порядков:
        %
\begin{itemize}
        %
\item
Числовые множества с обычным отношением порядка (здесь
порядок будет линейным).
        %
\item
На множестве $\mathbb{R}\hm\times\mathbb{R}$ всех пар действительных
чисел можно ввести частичный порядок, считая, что $\langle
x_1,x_2\rangle \hm\le \langle y_1,y_2\rangle$, если $x_1\hm\le y_1$
и~$x_2\hm\le y_2$. Этот порядок уже не будет линейным: пары
$\langle 0,1\rangle$ и~$\langle 1,0\rangle$ не сравнимы.
        %
\item
На множестве функций с действительными аргументами и значениями
можно ввести частичный порядок, считая, что $f\hm\le g$, если
$f(x)\hm\le g(x)$ при всех $x\in\bbR$. Этот порядок не будет
линейным.
        %
\item
На множестве целых положительных чисел можно определить порядок,
считая, что $x\hm\le y$, если $x$ делит~$y$. Этот порядок
тоже не будет линейным.
        %
\item
Отношение \лк любой простой делитель числа~$x$ является также и делителем
числа~$y$\пк\ не будет отношением порядка на множестве целых
положительных чисел (оно рефлексивно и транзитивно, но не
антисимметрично).
        %
\item
Пусть $U$\т произвольное множество. Тогда на множестве~$P(U)$ всех
подмножеств множества~$U$ отношение включения $\subset$ будет
частичным порядком.
        %
\item
\label{lexicographic-ordering}%
На буквах русского алфавита традиция определяет некоторый
порядок
($\text{а}\hm\le\text{б}\hm\le\text{в}\hm\le\ldots\hm\le\text{я}$).
Этот порядок линеен\т про любые две буквы можно сказать, какая из
них раньше (при необходимости заглянув в словарь).
        %
\item
На словах русского алфавита определён \emph{лексикографический}%
       \index{Лексикографический порядок}
порядок (как в словаре). Формально определить его можно так:
если слово~$x$ является началом слова~$y$, то $x\hm\le y$
(например, $\text{кант}\hm\le\text{кантор}$). Если ни одно из слов
не является началом другого, посмотрим на первую по порядку
букву, в которой слова отличаются: то слово, где эта буква
меньше в алфавитном порядке, и будет меньше. Этот порядок также
линеен (иначе что бы делали составители словарей?).
        %
\item
Отношение равенства ($(x\hm\le y)\hm\Leftrightarrow (x\hm=y)$)
также является отношением частичного порядка, для которого
никакие два различных элемента не сравнимы.
        %
\item
Приведём теперь бытовой пример. Пусть есть множество~$X$
картонных
коробок. Введём на нём порядок, считая, что
$x\hm\le
y$, если коробка~$x$ целиком помещается внутрь коробки~$y$ (или
если $x$~и~$y$\т одна и та же коробка). В зависимости от набора
коробок этот порядок может быть или не быть линейным.
        %
\end{itemize}

Пусть $x,y$\т элементы частично упорядоченного множества~$X$.
Говорят, что $x<y$\index{$a<b$}, если $x\le y$ и $x\ne y$. Для этого отношения
выполнены такие
свойства:
\vspace*{-1.4ex}
        \begin{gather*}
 x \not< x;\\
(x < y) \text{ и } (y < z)\ \Rightarrow\ x < z.
        \end{gather*}
(Первое очевидно, проверим второе: если $x\hm<y$ и~$y\hm<z$, то есть
$x\hm\le y$, $x\hm\ne y$, $y\hm\le z$, $y\hm\ne z$, то~$x\hm\le z$ по
транзитивности; если бы оказалось, что~$x\hm=z$, то мы бы имели
$x\hm\le y\hm\le x$ и потому~$x\hm=y$ по антисимметричности, что
противоречит предположению.)

Терминологическое замечание: мы читаем знак~$\le$ как \лк меньше
или равно\пк, а знак~$<$\т как \лк меньше\пк, неявно
подразумевая, что $x\hm\le y$ тогда и только тогда, когда $x\hm<y$ или~$x\hm=y$.
К счастью, это действительно так. Ещё одно замечание: выражение $x\hm>y$
(\лк $x$ больше~$y$\пк) означает, что~$y\hm<x$,
а выражение $x\ge y$ (\лк $x$ больше или равно~$y$\пк) означает,
что~$y\hm\le x$.

\begin{problem}
        %
Объясните, почему не стоит читать $x\hm\le y$ как \лк $x$ не больше~$y$\пк.
        %
\end{problem}

В некоторых книжках отношение частичного порядка определяется
как отношение~$<$, удовлетворяющее двум указанным свойствам. В
этом случае отношение $x\hm\le y \hm\Leftrightarrow [(x\hm<y)
\text{ или } (x\hm=y)]$ является отношением частичного порядка в
смысле нашего определения.

\begin{problem}
        %
Проверьте это.
        %
\end{problem}

Во избежание путаницы отношение $<$ иногда называют отношением
\emph{строгого порядка}, а отношение $\le$\т отношением
\emph{нестрогого порядка}. Одно и то же частично упорядоченное
множество можно задавать по\д разному: можно сначала определить
отношение нестрогого порядка~$\le$ (рефлексивное,
антисимметричное и транзитивное) и затем из него получить
отношение строгого порядка~$<$, а можно действовать и наоборот.

\begin{problem}
        %
Опуская требование антисимметричности в определении частичного порядка,
получаем определение \emph{предпорядка}\index{Предпорядок}.
Докажите, что любой
предпорядок устроен так: множество делится на непересекающиеся
классы, при этом $x\hm\le y$ для любых двух элементов
$x$, $y$ из одного класса, а на фактор\д множестве задан частичный
порядок, который и определяет результат сравнения двух элементов
из разных классов.
        %
\end{problem}

Вот несколько конструкций, позволяющих строить одни упорядоченные
множества из других.

\begin{itemize}
        %
\item
Пусть $Y$\т подмножество частично упорядоченного мно\-же\-с\-т\-ва~$(X,\le)$.
Тогда на множестве~$Y$ возникает естественный частичный порядок,
\emph{индуцированный}\index{Индуцированный порядок} из~$X$. Формально говоря,
        $$
(\le _Y) = (\le) \cap (Y\times Y).
        $$
Если порядок на~$X$ был линейным, то и индуцированный порядок на~$Y$,
очевидно, будет линейным.
        %
\item
Пусть $X$ и $Y$\т два непересекающихся частично
упорядоченных множества. Тогда на их объединении можно
определить частичный порядок так: внутри каждого множества
элементы сравниваются как раньше, а любой элемент
множества~$X$ по определению меньше любого элемента~$Y$. Это
множество естественно обозначить~$X+Y$\index{$A+B$}. (Порядок будет
линейным, если он был таковым на каждом из множеств.)

Это же обозначение применяют и для пересекающихся (и даже
совпадающих множеств). Например, говоря об упорядоченном
 множестве
$\mathbb{N}+\mathbb{N}$, мы берём две
непересекающиеся копии натурального ряда $\{0,1,2,\dots\}$ и
$\{\overline 0,\overline 1,\overline 2,\dots\}$ и рассматриваем множество
$\{0,1,2,\dots,\overline 0, \overline1, \overline2,\dots\}$, причём $k\le \overline l$
при всех $k$~и~$l$, а внутри каждой копии порядок обычный.
        %
\item
Пусть $(X, \le_{X})$ и $(Y,\le_Y)$\т два частично упорядоченных
множества. Можно определить порядок на произведении~$X\hm\times Y$ несколькими
способами. Можно считать, что $\langle x_1,y_1\rangle \hm\le
\langle x_2,y_2\rangle$, если $x_1 \le_X x_2$ и~$y_1\le_Y y_2$
(покоординатное сравнение). Этот порядок, однако, не будет
линейным, даже если исходные порядки и были линейными: если
первая координата больше у одной пары, а вторая у другой, как их
сравнить? Чтобы получить линейный
        \label{linear-cartesian-product}%
порядок, договоримся, какая координата будет \лк главной\пк\ и
будем сначала сравнивать по ней, а потом (в случае равенства)\т
по другой. Если главной считать $X$\д координату, то $\langle
x_1,y_1\rangle \hm\le\langle x_2,y_2\rangle$, если $x_1 <_X x_2$
или если $x_1=x_2$, а $y_1 \le_Y y_2$. Однако по техническим
причинам удобно считать главной вторую координату. Говоря о
произведении двух линейно упорядоченных множеств как о линейно
упорядоченном множестве, мы в дальнейшем подразумеваем именно
такой порядок (сначала сравниваем по второй координате).
        %
\end{itemize}

\begin{problem}
        %
Докажите, что в частично упорядоченном
множестве~$\mathbb{N}\hm\times\mathbb{N}$ (порядок покоординатный)
нет бесконечного
подмножества, любые два элемента
которого были бы несравнимы. Верно ли аналогичное
утверждение для~$\mathbb{Z}\hm\times\mathbb{Z}$?
        %
\end{problem}

\begin{problem}
        %
Докажите аналогичное утверждение для~$\mathbb{N}^k$ (порядок
покоординатный).
        %
\end{problem}

\begin{problem}
        %
Пусть $U$\т конечное множество из~$n$ элементов. Рассмотрим
множество~$P(U)$ всех подмножеств множества~$U$, упорядоченное
по включению. Какова максимально
возможная мощность множества $S\subset P(U)$, если
индуцированный на~$S$ порядок линеен? если никакие два
элемента~$S$ не сравнимы? (Указание: см.~задачу~\ref{noncomparable-subsets}.)
        %
\end{problem}

\begin{problem}
        %
Сколько существует различных линейных порядков на
множестве из~$n$ элементов?
        %
\end{problem}

\begin{problem}
        %
Докажите, что всякий частичный порядок на конечном множестве
можно продолжить до линейного (\лк продолжить\пк\ означает, что
если $x\hm\le y$ в исходном порядке, то и в новом это останется
так).
        %
\end{problem}

\begin{problem}
        %
Дано бесконечное частично упорядоченное множество~$X$. Докажите,
что в нём всегда найдётся либо бесконечное подмножество попарно
несравнимых элементов, либо бесконечное подмножество, на котором
индуцированный порядок линеен.
        %
\end{problem}

\begin{problem}
        %
(Конечный вариант предыдущей задачи.) Даны целые положительные числа
$m$~и~$n$. Докажите, что во всяком частично упорядоченном множестве
мощности~$mn+1$ можно указать либо $m+1$~попарно несравнимых
элементов, либо $n+1$~попарно сравнимых.
        %
\end{problem}

\begin{problem}
        %
В строчку написаны $mn+1$~различных чисел. Докажите, что можно
часть из них вычеркнуть так, чтобы осталась либо возрастающая
последовательность длины~$m+1$, либо убывающая
последовательность длины~$n+1$. (Указание: можно воспользоваться
предыдущей задачей.)
        %
\end{problem}

\begin{problem}
        %
Рассмотрим семейство всех подмножеств натурального ряда,
упорядоченное по включению. Существует ли у него
линейно упорядоченное (в индуцированном порядке) подсемейство
мощности континуум? Существует ли у него подсемейство
мощности континуум, любые два элемента которого несравнимы?
        %
\end{problem}

Элемент частично упорядоченного множества называют \emph{наибольшим}%
\index{Наибольший элемент},
если он больше
любого другого элемента, и \emph{максимальным}%
\index{Максимальный элемент},
если не существует большего элемента. Если множество не
является линейно упорядоченным, то это не одно и то же:
наибольший элемент
автоматически является максимальным,
но не наоборот. (Одно дело коробка, в которую помещается
любая другая, другое\т коробка, которая никуда больше
не помещается.)

Легко понять, что наибольший элемент в данном частично
упорядоченном множестве может быть только один, в
то время как максимальных элементов может быть много.

Аналогично определяют \emph{наименьшие}\index{Наименьший элемент}
 и \emph{минимальные}\index{Минимальный элемент} элементы.

\begin{problem}
        %
Докажите, что любые два максимальных элемента
не сравнимы. Докажите, что в конечном частично
упорядоченном множестве~$X$ для любого элемента~$x$
найдётся максимальный элемент~$y$, больший или равный~$x$.
        %
\end{problem}

\section{Изоморфизмы}\index{Изоморфизм}
        \label{isomorphisms}%

Два частично упорядоченных множества называются
\emph{изоморфными}\index{Множества!изоморфные}, если между ними существует
\emph{изоморфизм}\index{Изоморфизм}, то есть взаимно однозначное соответствие,
сохраняющее порядок. (Естественно, что в этом случае
они равномощны как множества.) Можно сказать так:
биекция $f\colon A\hm\to B$ называется изоморфизмом частично
упорядоченных множеств~$A$ и~$B$, если
        $$
 a_1 \le a_2 \ \Leftrightarrow \ f(a_1) \le f(a_2)
        $$
для любых элементов $a_1,a_2\hm\in A$ (слева знак~$\le$ обозначает
порядок в множестве~$A$, справа\т в множестве~$B$).

Очевидно, что отношение изоморфности рефлексивно (каждое множество
изоморфно самому себе), симметрично (если $X$~изоморфно~$Y$, то
и наоборот) и транзитивно (два множества, изоморфные третьему,
изоморфны между собой). Таким образом, все частично
упорядоченные множества разбиваются на классы изоморфных,
которые называют \emph{порядковыми типами}\index{Порядковый тип}.
(Правда, как и с
мощностями, тут необходима осторожность\т изоморфных множеств
слишком много, и потому говорить о порядковых типах как
множествах нельзя.)

\begin{theorem}\label{finite-linear-order}
        %
Конечные линейно упорядоченные множества из одинакового
числа элементов изоморфны.
        %
\end{theorem}

\begin{proof}
        %
Конечное линейно упорядоченное множество всегда имеет наименьший
элемент (возьмём любой элемент; если он не наименьший, возьмём меньший,
если и он не наименьший, ещё меньший\т и так далее; получим
убывающую последовательность $x\hm>y\hm>z\hm>\ldots$, которая рано или
поздно должна оборваться). Присвоим наименьшему элементу номер~$1$.
Из оставшихся снова выберем наименьший
элемент и присвоим ему номер~$2$ и так далее. Легко понять,
что порядок между элементами соответствует порядку между
номерами, то есть что наше множество изоморфно множеству
$\{1,2,\dots,n\}$.
        %
\end{proof}

\problskip

\begin{problem}
        %
Докажите, что множество всех целых положительных делителей
числа~$30$ с отношением \лк быть делителем\пк\ в качестве
отношения порядка изоморфно множеству всех подмножеств
множества~$\{a,b,c\}$, упорядоченному по включению.
        %
\end{problem}

\begin{problem}
        %
Будем рассматривать финитные последовательности%
\index{Финитные последовательности} натуральных
чисел, то есть последовательности, у которых все члены, кроме
конечного числа, равны~$0$. На множестве таких
последовательностей введём покомпонентный порядок:
$(a_0,a_1,\dots) \hm\le (b_0,b_1,\dots)$, если $a_i\hm\le b_i$
при всех~$i$. Докажите, что это множество изоморфно множеству
всех положительных целых чисел с отношением \лк быть
делителем\пк\ в качестве порядка.
        %
\end{problem}

\problskip

Взаимно однозначное отображение частично упорядоченного
множества~$A$ в себя, являющееся изоморфизмом,
называют
\emph{автоморфизмом}\index{Автоморфизм} частично
упорядоченного множества~$A$.
Тождественное отображение всегда является автоморфизмом, но
для некоторых множеств существуют и другие автоморфизмы. Например,
отображение прибавления единицы ($x\mapsto x+1$) является
автоморфизмом частично упорядоченного множества~$\bbZ$
целых чисел (с естественным порядком). Для множества
натуральных чисел та же формула не даёт автоморфизма
(нет взаимной однозначности).

\begin{problem}
        %
Покажите, что не существует автоморфизма упорядоченного множества~$\bbN$
натуральных чисел, отличного от тождественного.
        %
\end{problem}

\begin{problem}
        %
Рассмотрим множество $P(A)$ всех подмножеств некоторого
$k$\д элементного множества~$A$, частично упорядоченное
по включению. Найдите число автоморфизмов этого множества.
        %
\end{problem}

\begin{problem}
        %
Покажите, что множество целых положительных чисел, частично
упорядоченное отношением \лк $x$~делит~$y$\пк, имеет континуум
различных автоморфизмов.
        %
\end{problem}

Вот несколько примеров равномощных, но не изоморфных
линейно упорядоченных множеств (в
силу теоремы~\ref{finite-linear-order} они должны быть бесконечными).

\begin{itemize}
        %
\item
Отрезок~$[0,1]$ (с обычным отношением порядка)
не изоморфен множеству~$\bbR$, так как у первого
есть наибольший элемент, а у второго нет. (При изоморфизме
наибольший элемент, естественно, должен соответствовать
наибольшему.)
        %
\item
Множество~$\bbZ$ (целые числа с обычным порядком)
не изоморфно множеству~$\bbQ$ (рациональные числа).
В самом деле, пусть $\alpha\colon\bbZ\hm\to\bbQ$ является изоморфизмом.
Возьмём два соседних целых
числа, скажем, $2$ и~$3$.
При изоморфизме~$\alpha$ им должны соответствовать какие\д то два
рациональных числа $\alpha(2)$ и~$\alpha(3)$, причём
$\alpha(2)\hm<\alpha(3)$, так как~$2\hm<3$. Но тогда рациональным
числам между~$\alpha(2)$ и~$\alpha(3)$ должны соответствовать
целые числа между~$2$ и~$3$, которых нет.
        %
\item
Более сложный пример\т множества~$\bbZ$ и~$\bbZ\hm+\bbZ$.
Возьмём в~$\bbZ\hm+\bbZ$ две
копии нуля (из той и другой компоненты); мы обозначали их~$0$ и~$\overline 0$.
При этом~$0\hm<\overline 0$. При изоморфизме им должны
соответствовать два целых числа $a$ и~$b$, для
которых~$a\hm<b$. Тогда всем элементам между $0$ и~$\overline 0$ (их бесконечно
много: $1$, $2$, $3$, $\dots$, $-\overline3$, $-\overline2$, $-\overline1$)
должны соответствовать числа между $a$ и~$b$\т но их лишь конечное
число.

Этот пример принципиально отличается от предыдущих тем, что
здесь разницу между свойствами множеств нельзя записать
формулой. Как говорят,
упорядоченные множества~$\bbZ$
и~$\bbZ\hm+\bbZ$ \лк элементарно эквивалентны\пк\index{Элементарная
эквивалентность}.
        %
\end{itemize}

\begin{problem}
        %
Докажите, что линейно упорядоченные
множества~$\bbZ\hm\times\bbN$ и~$\bbZ\hm\times\bbZ$ (с
описанным выше на с.~\pageref{linear-cartesian-product} порядком) не
изоморфны.
        %
\end{problem}

\begin{problem}
        %
Будут ли изоморфны линейно упорядоченные
множества~$\bbN\hm\times\bbZ$ и~$\bbZ\hm\times\bbZ$?
        %
\end{problem}

\begin{problem}
        %
Будут ли изоморфны линейно упорядоченные
множества~$\bbQ\hm\times\bbZ$ и~$\bbQ\hm\times\bbN$?
        %
\end{problem}

Отображение~$x\hm\mapsto \sqrt{2}x$ осуществляет изоморфизм между
интервалами~$(0,1)$ и~$(0,\sqrt{2})$.
Но уже не так просто построить изоморфизм между
множествами рациональных точек этих интервалов (то есть
между~$\bbQ\hm\cap(0,1)$ и~$\bbQ\hm\cap(0,\sqrt{2})$), поскольку
умножение на~$\sqrt{2}$ переводит рациональные числа в
иррациональные. Тем не менее изоморфизм построить можно. Для
этого надо взять возрастающие последовательности рациональных чисел
$0\hm<x_1\hm<x_2\hm<\dots$ и $0\hm< y_1\hm<y_2\hm<\dots$, сходящиеся
соответственно к~$1$ и~$\sqrt{2}$, и построить кусочно\д линейную
функцию~$f$, которая переводит~$x_i$ в~$y_i$ и линейна на каждом
из отрезков~$[x_i,x_{i+1}]$ (рис.~\ref{iso-1}).
Легко понять, что она будет искомым
изоморфизмом.
        %
\begin{figure}[ht]
$$
\includegraphics{iso1-1.mps}
$$
\caption{Ломаная осуществляет изоморфизм.}
\label{iso-1}
\end{figure}
        %

\begin{problem}
        %
Покажите, что множество рациональных чисел интервала~$(0,1)$ и
множество~$\mathbb{Q}$ изоморфны. (Указание: здесь тоже можно
построить ломаную; впрочем, можно действовать иначе и начать с
того, что функция~$x\hm\mapsto 1/x$ переводит рациональные числа
в рациональные.)
        %
\end{problem}

\problskip
Более сложная конструкция требуется в следующей
задаче (видимо, ничего проще, чем сослаться на общую
теорему~\ref{dense-countable-sets}, тут не придумаешь).


\begin{problem}
        %
Докажите, что множество двоично\д рациональных чисел%
\index{Числа!двоично\д рациональные} интервала~$(0,1)$
изоморфно множеству~$\bbQ$. (Число считается
\emph{двоично\д рациональным}, если оно имеет вид~$m/2^n$, где
$m$\т целое число, а $n$\т натуральное.)
        %
\end{problem}

\problskip
Два элемента $x$, $y$ линейно упорядоченного множества называют
\emph{соседними}\index{Соседние элементы}, если~$x\hm<y$ и не
существует элемента между
ними, то есть такого~$z$, что~$x\hm<z\hm<y$. Линейно упорядоченное
множество называют \emph{плотным}\index{Множества!плотные},
если в нём нет соседних
элементов (то есть между любыми двумя есть третий).

\begin{theorem}\label{dense-countable-sets}
        %
Любые два счётных плотных линейно упорядоченных множества без
наибольшего и наименьшего элементов изоморфны.
        %
\end{theorem}

\begin{proof}
        %
Пусть $X$ и~$Y$\т данные нам множества. Требуемый изоморфизм
между ними строится по шагам. После $n$~шагов у нас есть два
$n$\д элементных подмножества $X_n\hm\subset X$ и~$Y_n\hm\subset Y$,
элементы которых мы будем называть \лк охваченными\пк, и
взаимно однозначное соответствие между ними, сохраняющее
порядок. На очередном шаге мы берём какой\д то неохваченный
элемент одного из
множеств (скажем, множества~$X$) и сравниваем его со всеми охваченными
элементами~$X$. Он может оказаться либо меньше всех,
либо больше, либо попасть между какими\д то двумя. В
каждом из случаев мы можем найти неохваченный элемент в~$Y$,
находящийся в том же положении (больше всех,
между первым и вторым
охваченным сверху,
между вторым и третьим
охваченным сверху \итп).
При этом мы пользуемся тем, что
в~$Y$ нет наименьшего элемента, нет наибольшего и нет соседних
элементов,\т в зависимости от того, какой из трёх случаев имеет
место. После этого мы добавляем выбранные элементы к $X_n$ и~$Y_n$,
считая их соответствующими друг другу.

Чтобы в пределе
получить изоморфизм между множествами $X$ и~$Y$, мы должны
позаботиться о том, чтобы все элементы обоих множеств были рано
или поздно охвачены. Это можно сделать так: поскольку каждое
из множеств счётно, пронумеруем его элементы и будем выбирать
неохваченный элемент с наименьшим номером (на нечётных шагах\т
из~$X$, на чётных\т из~$Y$). Это соображение завершает
доказательство.
        %
\end{proof}

\begin{problem}
        %
Сколько существует неизоморфных счёт\-ных плотных
линейно упорядоченных множеств (про наименьший и наибольший
элементы ничего не известно)? (Ответ:~$4$.)
        %
\end{problem}

\begin{problem}
        %
Приведите пример неизоморфных линейно упорядоченных множеств
мощности $\mathfrak{c}$ без наименьшего и наибольшего элементов.
(Указание: возьмите множества $\mathbb{Q}\hm+\mathbb{R}$
и~$\mathbb{R}\hm+\mathbb{Q}$.)
        %
\end{problem}

\begin{theorem}
        \label{Q-universal}
Всякое счётное линейно упорядоченное множество
изоморфно некоторому подмножеству множества~$\mathbb{Q}$.
        %
\end{theorem}

\begin{proof}
        %
Заметим сразу же, что вместо множества~$\mathbb{Q}$ можно было взять
любое плотное счётное всюду плотное множество без первого
и последнего элементов, так как они все изоморфны.

Доказательство этого утверждения происходит так же, как и в
теореме~\ref{dense-countable-sets}\т с той разницей, что
новые необработанные элементы берутся только с одной стороны
(из данного нам множества), а пары к ним подбираются в
множестве рациональных чисел.
        %
\end{proof}

\begin{problem}
        %
Дайте другое доказательство теоремы~\ref{Q-universal}, заметив,
что любое множество $X$ изоморфно подмножеству множества
$\bbQ\times X$.
        %
\end{problem}

\section{Фундированные множества}\index{Множества!фундированные}
        \label{well-founded-sets}
        \index{Фундированное множество}%

Принцип математической индукции\index{Индукция}
в одной из возможных форм звучит так:
        %
\quot{%
        %
Пусть $A(n)$\т некоторое свойство натурального числа~$n$. Пусть
нам удалось доказать~$A(n)$ в предположении, что $A(m)$ верно для
всех~$m$, меньших~$n$. Тогда свойство~$A(n)$ верно для
всех натуральных чисел~$n$.
        }

(Заметим, что по условию доказательство~$A(0)$ возможно без всяких
предположений, поскольку меньших чи\-сел~нет.)

Для каких частично упорядоченных множеств верен аналогичный
принцип? Ответ даётся следующей простой теоремой:

\begin{theorem}
        \label{well-founded-induction}%
Следующие три свойства частично упорядоченного множества~$X$
равносильны:

(\textsf{а}) любое непустое подмножество~$X$ имеет минимальный
элемент;

(\textsf{б}) не существует бесконечной строго убывающей
последовательности $x_0\hm>x_1\hm>x_2\hm>\dots$ элементов множества~$X$;

(\textsf{в}) для множества~$X$ верен принцип индукции в следующей
форме: если (при каждом~$x\hm\in X$)
из истинности~$A(y)$ для всех~$y\hm<x$ следует истинность~$A(x)$,
то свойство~$A(x)$ верно при всех~$x$.
Формально это записывают так:
        $$
\forall x\, (\forall y \, ((y<x)\Rightarrow A(y))\Rightarrow A(x))
\Rightarrow
\forall x\, A(x).
        $$
\end{theorem}

\begin{proof}
        %
Сначала докажем эквивалентность первых двух свойств. Если
$x_0\hm>x_1\hm>x_2>\dots$\т бесконечная убывающая последовательность,
то, очевидно, множество её значений не имеет минимального
элемента (для каждого элемента следующий ещё меньше).
Поэтому из~(а) следует~(б). Напротив, если $B$\т непустое
множество, не имеющее минимального элемента, то бесконечную
убывающую последовательность можно построить так. Возьмём
произвольный элемент~$b_0\hm\in B$. По предположению
он не является минимальным, так что можно найти~$b_1\hm\in B$,
для которого~$b_0\hm>b_1$. По тем же причинам можно найти $b_2\hm\in B$,
для которого~$b_1\hm>b_2$ \итд Получается бесконечная убывающая
последовательность.

Теперь выведем принцип индукции из существования минимального
элемента в любом подмножестве. Пусть $A(x)$\т произвольное
свойство элементов множества~$X$, верное не для всех элементов~$x$.
Рассмотрим непустое множество~$B$ тех элементов, для
которых свойство~$A$ неверно. Пусть $x$\т минимальный элемент
множества~$B$.
По условию меньших элементов в множестве~$B$ нет,
поэтому для всех~$y\hm<x$ свойство~$A(y)$ выполнено. Но тогда по
предположению должно быть выполнено и~$A(x)$\т
противоречие.

Осталось доказать существование минимального элемента в любом
непустом подмножестве, исходя из принципа индукции. Пусть $B$\т
подмножество без минимальных элементов. Докажем по
индукции, что $B$~пусто; другими словами, в качестве~$A(x)$
возьмём свойство~$x\hm\notin B$. В самом деле, если $A(y)$~верно
для всех~$y\hm<x$, то никакой элемент, меньший~$x$, не
лежит в~$B$. Если бы $x$~лежал в~$B$, то он
был бы там минимальным, а таких нет.
        %
\end{proof}

Множества, обладающие свойствами (а)\ч(в), называются
\emph{фундированными}.
        \index{Множества!фундированные}%
        \index{Фундированное множество}%
Какие есть примеры фундированных множеств? Прежде всего,
наш исходный пример\т множество натуральных чисел.

Другой пример\т множество~$\bbN\hm\times\bbN$
пар натуральных чисел (меньше та пара, у которой второй
член меньше; в случае равенства сравниваем первые).
В самом деле, проверим условие~(б). Нам будет удобно
сформулировать его так: всякая последовательность
        $
u_0\hm\ge u_1\hm\ge u_2 \ge\ldots
        $
элементов множества
рано или поздно стабилизируется (все члены, начиная с
некоторого, равны); очевидно, что это эквивалентная
формулировка.

Пусть дана произвольная последовательность пар
        $$
\langle x_0, y_0 \rangle \ge
\langle x_1, y_1 \rangle \ge
\langle x_2, y_2 \rangle \ge \dots
        $$
По определению порядка (сначала сравниваются вторые члены)
$y_0\hm\ge y_1\hm\ge y_2\ge\ldots$ и потому последовательность
натуральных чисел~$y_i$
с какого\д то места не меняется. После этого уже $x_i$~должны
убывать\т и тоже стабилизируются. Что и требовалось.

То же самое рассуждение пригодно и в более общей ситуации.
        %
\begin{theorem}
        %
Пусть $A$ и~$B$\т два фундированных частично упорядоченных
множества. Тогда их произведение~$A\hm\times B$, в котором
        $$
\langle a_1, b_1 \rangle \le
\langle a_2, b_2 \rangle \Leftrightarrow
[(b_1 < b_2) \text{ или } (b_1=b_2 \text{ и } a_1\le a_2)],
        $$
является фундированным.
        %
\end{theorem}

\begin{proof}
        %
В последовательности
$\langle a_0,b_0 \rangle \hm\ge
 \langle a_1,b_1 \rangle \ge\ldots$ стабилизируются сначала
вторые, а затем и первые члены.
        %
\end{proof}

Отсюда вытекает аналогичное утверждение
для~$\mathbb{N}\hm\times\mathbb{N}\times\mathbb{N}$, для~$\mathbb{N}^k$
или вообще для произведения конечного числа фундированных
множеств.

Ещё проще доказать, что сумма~$A+B$ двух фундированных множеств
$A$ и~$B$ фундирована: последовательность $x_0\ge x_1\ge
x_2\ge\dots$ либо целиком содержится в~$B$ (и мы ссылаемся на
фундированность~$B$), либо содержит элемент из~$A$. В последнем
случае все следующие элементы также принадлежат~$A$, и мы
используем фундированность~$A$.

\smallskip

Часто в программировании (или в олимпиадных задачах) нам нужно
доказать, что некоторый процесс не может продолжаться бесконечно
долго. Например, написав цикл, мы должны убедиться, что рано или
поздно из него выйдем. Это можно сделать так: ввести какой\д то
натуральный параметр и убедиться, что на каждом шаге цикла этот
параметр уменьшается. Тогда, если сейчас этот параметр равен~$N$,
то можно гарантировать, что не позже чем через $N$~шагов
цикл закончится.

Однако бывают ситуации, в которых число шагов заранее оценить
нельзя, но тем не менее гарантировать завершение цикла можно,
поскольку есть параметр, принимающий значения в фундированном
множестве и убывающий на каждом шаге цикла.

Вот пример олимпиадной задачи, где по существу такое рассуждение
и используется.

Бизнесмен заключил с чёртом сделку: каждый день он даёт
чёрту одну монету, и в обмен получает любой набор монет по
своему выбору, но все эти монеты меньшего достоинства (видов
монет конечное число). Менять (или получать) деньги в другом
месте бизнесмен не может. Когда монет больше не останется,
бизнесмен проигрывает.
    %   чёрту душу   (if any)
Докажите, что рано или поздно чёрт выиграет, каков бы ни был
начальный набор монет у бизнесмена.

Решение: пусть имеется $k$~видов монет. Искомый параметр
определим так: посчитаем, сколько монет каждого вида есть у
бизнесмена ($n_1$\т число монет минимального достоинства,
$n_2$\т число следующих, и так далее до~$n_k$). Заметим, что в
результате встречи с чёртом набор $\langle n_1,\dots,n_k\rangle$
уменьшается (в смысле введённого нами порядка, когда мы
сравниваем сначала последние члены, затем предпоследние \итд).
Поскольку множество~$\bbN^k$ фундировано, этот процесс должен
оборваться.

\problskip
\begin{problem}
        %
Имеется конечная последовательность нулей и единиц. За один
шаг разрешается сделать такое действие: найти в ней группу $01$
и заменить на $100{\dots}00$ (при этом можно написать сколько
угодно нулей). Докажите, что такие шаги нельзя выполнять бесконечно
много раз.
        %
\end{problem}

\begin{problem}
        %
Рассмотрим множество всех слов русского алфавита (точнее, всех
конечных последовательностей русских букв, независимо от смысла)
с лексикографическим порядком
        \index{Лексикографический порядок}%
(см.~с.~\pageref{lexicographic-ordering}). Будет ли это множество
фундировано?
        %
\end{problem}

\begin{problem}
        %
Рассмотрим множество невозрастающих последовательностей
натуральных чисел, в которых
все члены, начиная с некоторого, равны нулю. Введём
в нём порядок так: сначала сравниваем первые члены,
при равенстве первых вторые \итд Докажите, что
это (линейно) упорядоченное множество фундировано.
        %
\end{problem}

\begin{problem}
        %
Рассмотрим множество всех многочленов от одной переменной~$x$,
коэффициенты которых\т натуральные числа. Упорядочим его так:
многочлен~$P$ больше многочлена~$Q$, если $P(x)\hm>Q(x)$ для
всех достаточно больших~$x$. Покажите, что это определение
задаёт линейный порядок и что получающееся упорядоченное
множество фундировано.
        %
\end{problem}
