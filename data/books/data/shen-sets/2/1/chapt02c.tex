% !TEX root =  all.tex
\section{Ординалы}\index{Ординалы}
        \label{ordinals}

Как мы уже говорили, \emph{ординалом} называется порядковый тип
вполне упорядоченного множества, то есть класс всех изоморфных
ему упорядоченных множеств (естественно, они будут вполне
упорядоченными).

На ординалах естественно определяется линейный
порядок\index{Ординалы!сравнение}.
Чтобы сравнить два ординала~$\alpha$ и~$\beta$, возьмём их
представители~$A$ и~$B$.
Применим теорему~\ref{ordinals-trichotomy} и посмотрим, какой из
трёх случаев ($A$~изоморфно начальному
отрезку~$B$, отличному от всего~$B$; множества~$A$ и~$B$
изоморфны; $B$~изоморфно начальному отрезку~$A$, отличному от
всего~$A$) имеет место. В первом случае $\alpha\hm<\beta$,
во втором $\alpha\hm=\beta$, в
третьем~$\alpha\hm>\beta$.

Мы отвлекаемся от трудностей, связанных с основаниями теории
множеств (см.~раздел~\ref{cantor}); как формально можно
оправдать наши рассуждения, мы ещё обсудим. Пока что отметим
некоторые свойства ординалов.

\begin{itemize}
        %
\item
Мы определили на ординалах линейный порядок. Этот порядок будет
полным: любое непустое семейство ординалов имеет наименьший
элемент (теорема~\ref{complete-order-on-ordinals}; разница лишь
в том, что мы не употребляли там слова \лк ординал\пк, а
говорили о представителях).
        %
\item
Пусть $\alpha$\т некоторый ординал. Рассмотрим начальный отрезок~$[0,\alpha)$
в классе ординалов (образованный всеми ординалами,
меньшими~$\alpha$ в смысле
указанного порядка). Этот отрезок
упорядочен по типу~$\alpha$ (то есть изоморфен представителям
ординала~$\alpha$). В самом деле, пусть $A$\т один из
представителей ординала~$\alpha$. Ординалы, меньшие~$\alpha$,
соответствуют собственным (не совпадающим с~$A$) начальным
отрезкам множества~$A$. Такие отрезки имеют вид~$[0,a)$ и тем
самым находятся во взаимно однозначном соответствии с
элементами множества~$A$. (Легко проверить, что это соответствие
сохраняет порядок.)

Сказанное можно переформулировать так: каждый ординал упорядочен
как множество меньших ординалов. (В одном из формальных
построений теории ординалов каждый ординал \textsl{равен}
множеству всех меньших ординалов.)
        %
\item
Ординал называется \emph{непредельным}\index{Ординалы!непредельные},
если существует
непосредственно предшествующий ему (в смысле указанного порядка)
ординал. Если такого нет, ординал называют \emph{предельным}%
\index{Ординалы!предельные}.
        %
\item
Любое ограниченное семейство ординалов имеет точную верхнюю
грань (наименьший ординал, больший или равный всем ординалам
семейства). В самом деле, возьмём какой\д то ординал~$\beta$,
являющийся верхней границей. Тогда все ординалы семейства
изоморфны начальным отрезкам множества~$B$, представляющего
ординал~$\beta$. Если среди этих отрезков есть само~$B$, то
$\beta$~будет точной верхней гранью (и наибольшим элементом
семейства). Если нет, то эти отрезки имеют вид~$[0,b)$ для
различных элементов~$b\hm\in B$. Рассмотрим множество~$S$ всех
таких элементов~$b$. Если $S$~не ограничено в~$B$, то $\beta$~будет
точной верхней гранью. Если $S$~ограничено, то оно имеет
точную верхнюю грань~$s$, и $[0,s)$~будет точной верхней гранью
семейства.
        %
\end{itemize}

Можно сказать, что семейство ординалов\т это как бы
универсальное вполне упорядоченное семейство; любое
вполне упорядоченное множество изоморфно некоторому
начальному отрезку этого семейства. Поэтому
мы немедленно придём к противоречию, если захотим рассмотреть
множество всех ординалов (ведь для всякого вполне упорядоченного
множества есть ещё большее\т добавим к нему новый элемент,
больший всех предыдущих). Этот парадокс называется
\emph{парадоксом~Бурали\д Форти}\index{Парадокс!Бурали\д Форти}%
\glossary{Бурали-Форти@\Буралифорти}.

\begin{problem}
        %
Докажите, что точная верхняя грань счётного числа счётных
ординалов счётна.
        %
\end{problem}

Как же рассуждать об ординалах, не впадая в противоречия? В
принципе можно заменять утверждения об ординалах утверждениями о
их представителях и воспринимать упоминания ординалов как \лк
вольность речи\пк.
Другой подход (предложенный фон Нейманом)\glossary{Нейман@\фонНейман}
применяется при аксиоматическом
построении теории множеств, и состоит он примерно в следующем:
мы объявляем каждый ординал равным множеству всех меньших
ординалов. Тогда минимальный ординал~$0$ (порядковый тип пустого
множества) будет пустым множеством~$\varnothing$, следующий за
ним ординал~$1$ (порядковый тип одноэлементного множества) будет
$\{0\}\hm=\{\varnothing\}$, затем
$2\hm=\{0,1\}\hm=\{\varnothing, \{\varnothing\}\}$,
$3\hm=\{0,1,2\}\hm=\{\varnothing, \{\varnothing\},
\{\varnothing, \{\varnothing\}\}\}$,
$4\hm=\{0,1,2,3\}$~\итд
За ними следует ординал~$\omega$ (порядковый тип множества
натуральных чисел), равный $\{0,1,2,3,\dots\}$, потом
$\omega\hm+1\hm= \{0,1,2,3,\dots,\omega\}$, потом $\omega\hm+2\hm=
\{0,1,2,3,\dots,\omega,\omega+1\}$~\итд

Мы не будем говорить подробно об аксиоматической теории множеств%
\index{Аксиоматическая теория множеств}
Цермело\ч Френкеля\glossary{Цермело@\Цермело}\glossary{Френкель@\Френкель},
но два обстоятельства
следует иметь в виду.
Во\д первых, в ней нет никаких объектов, кроме множеств, и есть
\emph{аксиома экстенсиональности}\index{Аксиома!экстенсиональности
(объёмности)} (или \emph{объёмности}),
которая говорит, что два объекта, содержащие одни и те же
элементы, равны. Поэтому существует лишь один объект, не
содержащий элементов (пустое множество). Во\д вторых, в ней есть
\emph{аксиома фундирования}\index{Аксиома!фундирования}, которая говорит, что
отношение~$\in$ фундировано: во всяком множестве~$X$ есть
элемент, являющийся $\in$\д минимальным, то есть
элемент~$x\hm\in X$, для которого $X\hm\cap x\hm=\varnothing$.
Отсюда следует, что
никакое множество $x$ не может быть своим
элементом (иначе для множества~$\{x\}$ нарушалась бы аксиома фундирования).

\begin{problem}
        %
Выведите из аксиомы фундирования, что не существует множеств
$x$, $y$, $z$, для которых $x\in y \in z \in x$.
        %
\end{problem}

\problskip
Философски настроенный математик обосновал бы аксиому
фундирования так: множества строятся из ранее построенных
множеств, начиная с пустого, и поэтому возможна индукция по
построению (доказывая какое\д либо свойство множеств, можно
рассуждать индуктивно и предполагать, что оно верно для всех его
элементов).

Теперь можно определить ординалы так. Будем говорить, что
множество~$x$ \emph{транзитивно}\index{Транзитивное множество}%
\index{Множества!транзитивные}, если всякий элемент
множества~$x$ является подмножеством множества~$x$, то есть если
из $z\hm\in y\hm\in x$ следует~$z\hm\in x$. Назовём
\emph{ординалом}\index{Ординалы} транзитивное множество, всякий элемент которого
транзитивен. Это требование гарантирует, что на элементах любого
ординала отношение~$\in$ является (строгим)
частичным порядком.

Аксиома фундирования гарантирует, что частичный порядок~$\in$ на
любом ординале является фундированным. После этого
по индукции можно доказать, что он является линейным (и,
следовательно, полным).

\problskip
\begin{problem}
        %
(\textbf{а})~%
Используя определение ординала
как транзитивного множества с транзитивными элементами,
докажите, что элемент ординала есть ординал.
        %
(\textbf{б})~%
Пусть $\alpha$\т ординал (в смысле данного нами определения).
Докажите, что отношение $\in$ на нём является частичным порядком.
        %
(\textbf{в})~%
Докажите, что для любых элементов $a,b\hm\in\alpha$ верно ровно
одно из трёх соотношений: либо $a\hm\in b$, либо $a\hm=b$, либо $b\hm\in
a$. (Указание: используйте двойную
индукцию по фундированному отношению~$\in$
на~$\alpha$, а также аксиому экстенсиональности.)
        %
(\textbf{г})~%
Докажите, что один ординал изоморфен собственному начальному
отрезку другого тогда и только тогда, когда является его элементом.
(Таким образом, отношение~$<$ на ординалах как упорядоченных
множествах совпадает с отношением принадлежности.) Докажите, что
каждый ординал является множеством всех меньших его ординалов.
        %
\end{problem}

\problskip
Заметим ещё, что если каждый ординал есть множество
всех меньших его ординалов, то точная верхняя грань множества
ординалов есть их объединение.

Мы не будем подробно развивать этот подход и по\д
прежнему будем наивно представлять себе ординалы как порядковые
типы вполне упорядоченных множеств.

Прежде чем перейти к сложению и умножению ординалов,
отметим такое свойство (уже упомянутое в задаче~\ref{subset-initial}):

\begin{theorem}
        \label{subset-less-or-equal}
Пусть $A$\т подмножество вполне упорядоченного множества~$B$.
Тогда порядковый\index{Порядковый тип} тип множества~$A$ не превосходит
порядкового
типа множества~$B$.
        %
\end{theorem}

\begin{proof}
        %
Отметим сразу же, что равенство возможно, даже если $A$~является
собственным подмножеством~$B$. Например, чётные натуральные
числа имеют тот же порядковый тип~$\omega$, что и все
натуральные числа.

Рассуждая от противного, предположим, что порядковый тип
множества~$A$ больше. Тогда $B$~изоморфно некоторому начальному
отрезку множества~$A$, не совпадающему со всем~$A$. Пусть $a_0$\т
верхняя граница (в~$A$) этого отрезка, а $f\colon B \hm\to A$\т
соответствующий изоморфизм. Тогда $f$~строго возрастает и
потому $f(b)\hm\ge b$ для всех~$b\hm\in B$
(теорема~\ref{monotone-mappings-wosets}). В частности, $f(a_0)\hm\ge a_0$,
но по предположению любое значение~$f(b)$ меньше~$a_0$\т
противоречие.
        %
\end{proof}

\section{Арифметика ординалов}
        \label{ordinal-arithmetics}

Мы определили сумму\index{Сумма множеств}\index{Множества!сумма}
и произведение\index{Множества!произведение}\index{Произведение множеств}
линейно упорядоченных
множеств в разделе~\ref{equivalence-order}. (Напомним, что
в~$A\hm+B$ элементы~$A$ предшествуют элементам~$B$, а в $A\hm\times B$
мы сначала сравниваем $B$\д компоненты пар, а в случае их равенства\т
$A$\д
компоненты.)

Легко проверить следующие свойства сложения\index{Сложение ординалов}:
        %
\begin{itemize}
        %
\item Сложение ассоциативно\index{Ассоциативность}:
$\alpha\hm+(\beta\hm+\gamma)\hm=(\alpha\hm+\beta)\hm+\gamma$.
        %
\item Сложение не коммутативно\index{Коммутативность}: например,
$1\hm+\omega\hm=\omega$, но
$\omega\hm+1\hm\ne\omega$.
        %
\item Очевидно, $\alpha\hm+0\hm=0\hm+\alpha\hm=\alpha$.
        %
\item Сумма возрастает при росте
второго аргумента: если
$\beta_1\hm<\beta_2$, то $\alpha\hm+\beta_1\hm<\alpha\hm+\beta_2$.
(В самом деле, пусть $\beta_1$~изоморфно начальному отрезку в~$\beta_2$,
отличному от всего~$\beta_2$.  Добавим к этому изоморфизму
тождественное отображение на~$\alpha$ и получим изоморфизм
между~$\alpha\hm+\beta_1$ и начальным отрезком в~$\alpha\hm+\beta_2$, отличным
от~$\alpha\hm+\beta_2$.)
        %
\item Сумма неубывает при росте первого аргумента: если
$\alpha_1\hm<\alpha_2$, то $\alpha_1\hm+\beta \hm\le
\alpha_2\hm+\beta$. (В самом деле, $\alpha_1\hm+\beta$~изоморфно
подмножеству в~$\alpha_2\hm+\beta$. Это подмножество не является
начальным отрезком, но мы можем воспользоваться
теоремой~\ref{subset-less-or-equal}.)
        %
\item
Определение суммы согласовано с обозначением~$\alpha\hm+1$ для следующего
за~$\alpha$ ординала. (Здесь~$1$\т порядковый тип одноэлементного
множества.) Следующим за~$\alpha\hm+1$ ординалом будет ординал
$(\alpha\hm+1)\hm+1\hm=\alpha\hm+(1\hm+1)\hm=\alpha\hm+2$~\итд
        %
\item Если $\alpha\hm\ge\beta$, то существует единственный
ординал\index{Ординалы}~$\gamma$,
для которого $\beta\hm+\gamma\hm=\alpha$. (В самом деле, $\beta$~изоморфно
начальному отрезку в~$\alpha$; оставшаяся часть~$\alpha$ и будет
искомым ординалом~$\gamma$. Единственность следует из монотонности
сложения по второму аргументу.) Заметим, что эту операцию можно называть
\лк вычитанием\index{Вычитание ординалов} слева\пк.
        %
\item
\лк Вычитание справа\пк, напротив, возможно не всегда. Пусть~$\alpha$\т
некоторый ординал. Тогда уравнение $\beta\hm+1\hm=\alpha$ (относительно~$\beta$)
имеет решение тогда и только тогда, когда $\alpha$\т непредельный ординал,
(\те~когда $\alpha$~имеет наибольший элемент).
        %
\end{itemize}

Определение суммы двух ординалов в силу ассоциативности можно
распространить на любое конечное число ординалов. Можно
определить и сумму $\alpha_1\hm+\alpha_2\hm+\ldots$ счётной
последовательности ординалов (элементы~$\alpha_i$ предшествуют
элементам~$\alpha_j$ при~$i\hm<j$; внутри каждого~$\alpha_i$
порядок прежний). Как легко проверить, это
множество действительно
будет вполне упорядоченным: чтобы найти минимальный элемент в его
подмножестве, рассмотрим компоненты, которые это подмножество
задевает, выберем из них компоненту с наименьшим номером и
воспользуемся её полной упорядоченностью.

В этом построении можно заменить натуральные числа на элементы
произвольного вполне упорядоченного множества~$I$ и определить
сумму $\sum A_i$ семейства вполне упорядоченных
множеств~$A_i$, индексированного элементами~$I$, как порядковый
тип множества всех пар вида $\langle a,i\rangle$, для которых~$a\in A_i$.
При сравнении пар сравниваются вторые компоненты, а в случае
равенства и первые (в соответствующем~$A_i$). Если все~$A_i$
изоморфны одному и тому же множеству~$A$, получаем уже известное
нам определение произведения~$A\hm\times I$.

\medskip
Теперь перейдём к умножению ординалов\index{Умножение ординалов}.
        %
\begin{itemize}
        %
\item Умножение ассоциативно\index{Ассоциативность}: $(\alpha\beta)\gamma\hm=
\alpha(\beta\gamma)$. (В самом деле, в обоих случаях
по существу получается множество троек; тройки сравниваются
справа налево, пока не обнаружится различие.)
        %
\item Умножение не коммутативно\index{Коммутативность}: например, $2\hm\cdot\omega\hm=\omega$,
в то время как $\omega\hm\cdot 2\hm\ne\omega$.
        %
\item Очевидно, $\alpha\hm\cdot0\hm=0\hm\cdot\alpha=0$ и
$\alpha\cdot1\hm=1\cdot\alpha\hm=\alpha$.
        %
\item Выполняется одно из свойств дистрибутивности\index{Дистрибутивность}:
$\alpha(\beta\hm+\gamma)\hm=\alpha\beta\hm+\alpha\gamma$
(непосредственно следует из определения). Симметричное свойство
выполнено не всегда: $(1\hm+1)\hm\cdot\omega\hm=\omega\hm\ne\omega+\omega$.
        %
\item Произведение строго возрастает при увеличении второго
множителя, если первый не равен~$0$. (Для разнообразия выведем
это из ранее доказанных свойств: если $\beta_2\hm>\beta_1$, то
$\beta_2\hm=\beta_1\hm+\delta$, так что
$\alpha\beta_2\hm=\alpha(\beta_1\hm+\delta)
\hm=\alpha\beta_1 \hm+\alpha\delta
\hm>\alpha\beta_1$.)
        %
\item Произведение не убывает при возрастании первого множителя.
(В самом деле, если $\alpha_1\hm<\alpha_2$, то
$\alpha_1\beta$ изоморфно
подмножеству~$\alpha_2\beta$. Это подмножество не является
начальным отрезком, но можно сослаться на
теорему~\ref{subset-less-or-equal}.)
        %
\label{ordinal-product-structure}%
\item Любой ординал, меньший $\alpha\beta$, однозначно
представим в виде $\alpha\beta'\hm+\alpha'$, где $\beta'\hm<\beta$
и~$\alpha'\hm<\alpha$.

(В самом деле, пусть множества $A$
и~$B$ упорядочены по типам $\alpha$ и~$\beta$. Тогда $A\hm\times B$
упорядочено по типу $\alpha\beta$. Всякий ординал, меньший
$\alpha\beta$, есть начальный отрезок в $A\hm\times B$,
ограниченный некоторым элементом~$\langle a,b\rangle$.
Начальный отрезок~$[0,\langle a,b\rangle)$ состоит из пар, у
которых второй член меньше~$b$, а также из пар, у которых второй
член равен~$b$, а первый меньше~$a$. Отсюда следует, что этот
начальный отрезок изоморфен $A\hm\times[0,b)\hm+[0,a)$, так что
остаётся положить $\beta'\hm=[0,b)$ и $\alpha'\hm=[0,a)$. Теперь
проверим однозначность. Пусть
$\alpha\beta'\hm+\alpha'\hm=\alpha\beta''\hm+\alpha''$. Если
$\beta'=\beta''$, то можно воспользоваться однозначностью левого
вычитания и получить, что $\alpha'=\alpha''$. Остаётся
проверить, что $\beta'$ не может быть, скажем, меньше $\beta''$.
В этом случае $\beta''\hm=\beta'+\delta$, и сокращая $\alpha\beta'$
слева, получим, что $\alpha'\hm=\alpha\delta+\alpha''$, что
невозможно, так как левая часть меньше~$\alpha$, а правая часть
больше или равна~$\alpha$.)
        %
\item Аналогичное \лк деление с остатком\index{Деление ординалов
с остатком}\пк\ возможно и для
любых ординалов. Пусть~$\alpha\hm>0$. Тогда любой ординал~$\gamma$
можно разделить с остатком на~$\alpha$, то есть представить в
виде $\alpha\tau\hm+\rho$, где~$\rho\hm<\alpha$, и притом
единственным образом.

(В самом деле, существование следует из предыдущего утверждения,
надо только взять достаточно большое~$\beta$, чтобы $\alpha\beta$
было больше~$\gamma$, скажем, $\beta\hm=\gamma\hm+1$. Единственность
доказывается так же, как и в предыдущем пункте.)
        %
\item
\label{position-ordinal-system}%
Повторяя деление с остатком на $\alpha>0$, можно построить
позиционную систему счисления\index{Позиционная система счисления} для
ординалов: всякий ординал,
меньший $\alpha^{k+1}$ (здесь~$k$\т натуральное число), можно
однозначно представить в виде
$\alpha^k\beta_k\hm+\alpha^{k-1}\beta_{k-1}\hm+
\ldots\hm+\alpha\beta_1\hm+\beta_0$, где \лк цифры\пк\
$\beta_k$, $\dots$, $\beta_1$, $\beta_0$\т ординалы, меньшие~$\alpha$.

\end{itemize}

\begin{problem}
        %
Для каких ординалов~$1\hm+\alpha\hm=\alpha$?
        %
\end{problem}

\begin{problem}
        %
Для каких ординалов~$2\hm\cdot\alpha\hm=\alpha$?
        %
\end{problem}

\begin{problem}
        %
Какие ординалы представимы в виде~$\omega\hm\cdot\alpha$?
        %
\end{problem}

\begin{problem}
        %
Докажите, что ${\alpha+\beta}\hm=\beta$ тогда и только тогда,
когда ${\alpha\omega}\hm\le\beta$ (здесь~$\alpha$ и~$\beta$\т
ординалы).
        %
\end{problem}

\begin{problem}
        %
Докажите, что если ${\alpha+\beta}\hm={\beta+\alpha}$ для некоторых
ординалов~$\alpha$ и~$\beta$, то найдётся такой ординал~$\gamma$ и
такие натуральные числа~$m$ и~$n$, что $\alpha\hm={\gamma m}$ и
$\beta\hm={\gamma n}$.
        %
\end{problem}

\begin{problem}
        %
Определим операцию \лк замены основания\пк\ с~$k\hm>1$ на~$l\hm>k$.
Чтобы применить эту операцию к натуральному числу~$n$, надо
записать~$n$ в $k$\д ичной системе счисления, а затем
прочесть эту запись в $l$\д ичной системе. (Очевидно, число при этом
возрастёт, если оно было больше или равно~$k$.)
Возьмём произвольное число~$n$ и будет выполнять над ним
такие операции: замена основания с~$2$ на~$3$~-- вычитание единицы~--
замена основания с~$3$ на~$4$~-- вычитание единицы~--
замена основания с~$4$ на~$5$~-- вычитание единицы~-- \dots
Докажите, что рано или поздно мы получим нуль и вычесть единицу
не удастся. (Указание: замените все основания на ординал~$\omega$;
получится убывающая последовательность ординалов.)
        %
\end{problem}

\section{Индуктивные определения и степени}

Мы определили сложение и умножение ординалов с помощью
явных конструкций порядка на соответствующих множествах.
Вместо этого можно было бы их определить индуктивно.

\begin{theorem}
        \label{inductive-definition-sum}
Сложение ординалов\index{Сложение ординалов!индуктивное определение}
обладает следующими свойствами:
        \begin{align*}
\alpha+ 0 &= \alpha;\\
\alpha+(\beta+1) &= (\alpha+\beta)+1;\\
\alpha+\gamma&= \sup\{\alpha+\beta\mid \beta<\gamma\}
    \text{ для предельного $\gamma\hm\ne0$}.
        \end{align*}
Эти свойства однозначно определяют операцию сложения.
        %
\end{theorem}

\begin{proof}
        %
Два первых свойства очевидны; проверим третье. Если~$\beta\hm<\gamma$,
то $\alpha\hm+\beta\hm<\alpha\hm+\gamma$, так что $\alpha\hm+\gamma$~будет
верхней
границей всех сумм вида~$\alpha\hm+\beta$ при $\beta\hm<\gamma$.
Надо проверить, что эта граница точная. Пусть некоторый ординал~$\tau$
меньше $\alpha\hm+\gamma$. Убедимся, что он меньше~$\alpha\hm+\beta$
для некоторого~$\beta\hm<\gamma$. Если~$\tau\hm<\alpha$, всё очевидно.
Если~$\tau\hm\ge\alpha$, представим его в виде $\tau\hm=\alpha\hm+\sigma$.
Тогда $\alpha\hm+\sigma \hm< \alpha\hm+\gamma$ и потому~$\sigma\hm<\gamma$.
Поскольку ординал~$\gamma$ предельный, $\sigma\hm+1$~также меньше~$\gamma$
и остаётся положить~$\beta\hm=\sigma\hm+1$.

Указанные свойства однозначно определяют операцию сложения, так
как представляют собой рекурсивное определение по~$\beta$ (если
есть две операции сложения, обладающие этими свойствами, возьмём
минимальное~$\beta$, для которого они различаются~\итд).
        %
\end{proof}

Аналогично можно определить и умножение:

\begin{theorem}
        \label{inductive-definition-product}
Умножение ординалов\index{Умножение ординалов!индуктивное определение} обладает
следующими свойствами:
        \begin{align*}
\alpha 0 &= 0;\\
\alpha(\beta+1) &= \alpha\beta+\alpha;\\
\alpha\gamma&= \sup\{\alpha\beta\mid \beta<\gamma\}
    \text{ для предельного $\gamma\hm\ne0$}.
        \end{align*}
Эти свойства однозначно определяют операцию умножения.
        %
\end{theorem}

\begin{proof}
        %
Доказательство аналогично, нужно только проверить, что
если $\tau\hm<\alpha\gamma$ для предельного~$\gamma$, то
$\tau\hm<\alpha\beta$ для некоторого~$\beta\hm<\gamma$.
Как мы видели на с.~\pageref{ordinal-product-structure},
ординал~$\tau$ имеет вид $\tau\hm=\alpha\gamma'\hm+\alpha'$
при~$\gamma'\hm<\gamma$; достаточно положить~$\beta\hm=\gamma'\hm+1$.
        %
\end{proof}

Возникает естественное желание определить операцию
возведения в степень\index{Возведение ординалов в степень}. Мы уже по
существу определили
возведение в целую положительную степень ($\alpha^n$~есть
произведение $n$~сомножителей, равных~$\alpha$).
Другими словами, если $A$~упорядочено по
типу~$\alpha$, то множество~$A^n$ последовательностей длины~$n$
с элементами из~$A$ с обратным лексикографическим порядком
(сравнение справа налево) упорядочено по типу~$\alpha^n$.

Следующий шаг\т определить~$\alpha^\omega$. Первая идея,
приходящая в голову\т взять множество $A^{\bbN}$ бесконечных
последовательностей и определить на нём полный порядок. Но
как его ввести\т неясно. Поэтому можно попробовать
определить \emph{возведение в степень}%
\index{Возведение ординалов в степень!индуктивное определение} индуктивно
с помощью
следующих соотношений:
        \begin{align*}
\alpha^0 &= 1;\\
\alpha^{\beta+1} &= \alpha^\beta\cdot\alpha;\\
\alpha^\gamma&= \sup\{\alpha^\beta\mid \beta<\gamma\}
    \text{ для предельного $\gamma\hm\ne0$}.
        \end{align*}
Теорема~\ref{transfinite-recursion-theorem} (о трансфинитной
рекурсии) гарантирует, что эти соотношения однозначно определяют
некоторую операцию над ординалами, которая и называется
возведением в степень.

\textsf{Замечание.} Тут опять мы подходим к опасной границе
парадоксов и вынуждены выражаться уклончиво. На самом деле
теорема о трансфинитной рекурсии говорила об определении функции
на вполне упорядоченном множестве, а ординалы не образуют
множества\т их слишком много. Кроме того, в ней шла речь о
функциях со значениями в некотором заданном множестве, которого
здесь тоже нет. Подобные индуктивные определения можно корректно
обосновать в теории множеств с использованием так называемой
\emph{аксиомы подстановки}\index{Аксиома!подстановки}, но мы об этом
говорить не будем. Вместо
этого мы дадим явное описание возведения в степень, свободное от
этих проблем.

Чтобы понять смысл возведения в степень, посмотрим, как выглядит
ординал~$\alpha^\omega$ (для некоторого~$\alpha$). Пусть $A$\т
множество, упорядоченное по типу~$\alpha$. Ординал~$\alpha^\omega$
по определению есть точная верхняя грань~$\alpha^n$ для
натуральных~$n$. Ординал~$\alpha^n$ есть
порядковый тип множества~$A^n$, упорядоченного в обратном
лексикографическом порядке. Чтобы найти точную верхнюю грань,
представим множества~$A^n$ как начальные отрезки друг друга.
Например, $A^2$~состоит из пар~$\langle a_1,a_2\rangle$ и
отождествляется с начальным отрезком в~$A^3$, состоящим из
троек~$\langle a_1,a_2,0\rangle$. (Здесь $0$\т наименьший элемент
в~$A$.) Теперь видно, что все множества~$A^n$ можно рассматривать как
начальные отрезки множества~$A^\infty$, состоящего из бесконечных
последовательностей $a_0, a_1, \dots$, элементы которых
принадлежат~$A$ и в которых лишь конечное число членов отлично
от нуля. (Последнее требование делает корректным определение
обратного лексикографического порядка\т мы находим самую правую
позицию, в которой последовательности различаются, и сравниваем
их значения в этой позиции.) В объединении эти начальные отрезки
дают всё~$A^\infty$, так что это множество с описанным порядком
имеет тип~$\alpha^\omega$.

Аналогичным образом можно описать возведение в произвольную
степень.

Пусть $A$ и $B$\т вполне упорядоченные множества,  имеющие порядковые
типы~$\alpha$ и~$\beta$. Рассмотрим множество~$[B\to A]$,\index{$[B\to A]$}
состоящее из отображений~$B$ в~$A$, имеющих \лк конечный
носитель\index{Носитель отображения}\пк\ (равных минимальному элементу~$A$ всюду, за
исключением конечного множества). Введём на~$[B\to A]$
порядок: если~$f_1\hm\ne f_2$, выберем наибольший элемент~$b\hm\in B$,
для которого $f_1(b)\hm\ne f_2(b)$ и сравним $f_1(b)$ и~$f_2(b)$.

\begin{theorem}
        \label{ordinal-power-structure}
        %
Указанное правило задаёт полный порядок на множестве~$[B\to A]$
и порядковый тип этого множества есть~$\alpha^\beta$.
        %
\end{theorem}

\begin{proof}
        %
Нам надо проверить, что указанный порядок является
полным и что выполнены требования индуктивного определения
степени.

Назовём \emph{носителем}\index{Носитель отображения}
элемента~$f\hm\in[B\to A]$ множество
тех~$b\hm\in B$, для которых~$f(b)\hm>0$ (здесь $0$~обозначает
наименьший элемент множества~$A$). Назовём \emph{рангом}
функции~$f$ наибольший элемент носителя (по определению носитель
конечен, так что наибольший элемент существует). Ранг определён
для всех функций, кроме тождественно нулевой, которая является
минимальным элементом множества $[B\to A]$. Чем больше ранг
функции, тем больше сама функция в смысле введённого нами
порядка.

Пусть порядок на $[B\to A]$ не является полным и
$f_0\hm>f_1\hm>f_2\hm>\ldots$\т убывающая последовательность
элементов~$[B\to A]$. Все элементы~$f_i$ отличны от~$0$;
рассмотрим их ранги. Эти ранги образуют невозрастающую
последовательность, поэтому начиная с некоторого места
стабилизируются (множество~$B$ вполне упорядочено). Отбросим
начальный отрезок и будем считать, что с самого начала ранги
всех элементов убывающей последовательности одинаковы и равны
некоторому~$b$. В соответствии с определением, значения $f_0(b),
f_1(b),\dots$ образуют невозрастающую последовательность,
поэтому начиная с некоторого места стабилизируются. Отбросив
начальный отрезок, будем считать, что все~$f_i$ имеют одинаковый
ранг~$b$ и одинаковое значение~$f_i(b)$. Тогда значения~$f_i(b)$
не влияют на сравнения, и потому их можно заменить на~$0$.
Получим убывающую последовательность элементов~$[B\to A]$ с
рангами меньше~$b$. Чтобы завершить рассуждение, остаётся
сослаться на принцип индукции по множеству~$B$.

(Более формально, рассмотрим все бесконечно убывающие
последовательности. У каждой из них рассмотрим ранг первого
элемента. Рассмотрим те из них, у которых этот ранг минимально
возможный; пусть $b$\т это минимальное значение. В любой такой
последовательности все элементы имеют ранг~$b$. Из всех таких
последовательностей $f_0\hm>f_1\hm>\ldots$ выберем ту, у которой
значение~$f_0(b)$ минимально; все следующие её члены имеют то же
значение в точке~$b$ (\те~$f_i(b)\hm=f_0(b)$). Заменив значение в
точке~$b$ нулём, получим бесконечную убывающую
последовательность из элементов меньшего ранга, что противоречит
предположению.)

Теперь покажем, что такое явное определение степени согласовано
с индуктивным определением. Для конечных~$n$ это очевидно.
Пусть~$\gamma\hm=\beta\hm+1$. Каково (явное) определение~$\alpha^\gamma$?
Пусть $B$~упорядочено по типу~$\beta$. Тогда мы должны добавить
к~$B$ новый наибольший элемент (обозначим его~$m$) и
рассмотреть отображения $B\cup\{m\}\to A$ с конечным носителем.
Ясно, что такое отображение задаётся парой, состоящей из его
сужения на~$B$ (которое может быть произвольным элементом
множества~$[B\to A]$) и значения на~$m$. При определении порядка
мы сначала сравниваем значения на~$m$, а потом сужения на~$B$,
то есть полученное множество изоморфно~$[B\to A]\hm\times A$, что
и требовалось.

Пусть теперь $\gamma$\т ненулевой предельный ординал и множество~$C$ упорядочено
по типу~$\gamma$. Как устроено множество~${[C\to A]}$?
Элементы, ранг которых меньше~$c\hm\in C$, образуют в нём
начальный отрезок, и этот начальный отрезок изоморфен~$[[0,c)\to A]$.
А само множество~$[C\to A]$ является
объединением этих начальных отрезков (поскольку каждый элемент этого множества
имеет конечный носитель) и потому его
порядковый тип является точной верхней гранью их
порядковых типов, что и требовалось.
        %
\end{proof}

Непосредственным следствием этой теоремы является такое утверждение:
        %
\begin{theorem}
        %
Если $\alpha$ и~$\beta$\т счётные ординалы, то
$\alpha+\beta$, $\alpha\beta$ и~$\alpha^\beta$ счётны.
        %
\end{theorem}

\begin{proof}
        %
Для суммы и произведения утверждение очевидно.
Для степени: если мы пронумеровали все элементы вполне упорядоченных
множеств $A$ и~$B$, то любой элемент множества~$[B\to A]$
может быть задан конечным списком натуральных
чисел (носитель и значения на элементах носителя),
а таких списков счётное число.
        %
\end{proof}

\problskip

\begin{problem}
        %
Докажите, что $\alpha^{\beta+\gamma}=\alpha^\beta\cdot\alpha^\gamma$ двумя
способами: по индукции и с использованием явного определения степени.
        %
\end{problem}

\begin{problem}
        %
Докажите, что $(\alpha^\beta)^\gamma=\alpha^{\beta\gamma}$.
        %
\end{problem}

\begin{problem}
        %
Докажите, что если $\alpha\ge2$, то $\alpha^\beta\ge\alpha\beta$.
        %
\end{problem}

\begin{problem}
        %
Докажите, что если $\omega^\gamma=\alpha+\beta$ для некоторых
ординалов $\alpha$, $\beta$ и $\gamma$, то либо $\beta=0$, либо
$\beta=\omega^\gamma$.
        %
\end{problem}

\begin{problem}
        %
Какие ординалы нельзя представить в виде суммы двух меньших ординалов?
        %
\end{problem}

\begin{problem}
        %
Докажите счётность~$\alpha^\beta$ для счётных $\alpha$ и~$\beta$,
используя индуктивное определение степени.
        %
\end{problem}

\begin{problem}
        %
Дан некоторый ординал~$\alpha\hm>1$. Укажите наименьший
ординал~$\beta\hm>0$, для которого $\alpha\beta\hm=\beta$.
(Указание: что будет, если умножить~$x$ на степенной
ряд $1\hm+x\hm+x^2\hm+x^3\hm+\ldots$?)
        %
\end{problem}

\problskip
Отметим важную разницу между возведением ординалов
в степень и ранее рассмотренными операциями сложения и умножения
ординалов.
    \index{Возведение в степень мощностей}%
    \index{Мощности!возведение в степень}%
Определяя сумму и произведение ординалов,
мы вводили некоторый
порядок на сумме и произведении соответствующих
множеств (в
обычном смысле), здесь же само множество~${[B\to A]}$ определяется с
учётом порядка и отлично от~$A^B$. (В частности,
при счётных~$A$ и~$B$ множество
${[B\hm\to A]}$ счётно, а~$A^B$\т нет.)

\smallskip

Явное описание множества~$[B\to A]$ позволяет понять, как
устроены его начальные отрезки, то есть какой вид имеют
ординалы, меньшие~$\alpha^\beta$.

Рассмотрим некоторую функцию~$f\hm\in [B\to A]$. Пусть она
отлична от нуля в точках $b_1\hm>b_2\hm>\ldots\hm>b_k$ и принимает
в этих точках значения $a_1$, $a_2$, $\dots$, $a_k$. Нас интересуют все
функции, меньшие функции~$f$.

Все они равны нулю в точках, больших~$b_1$. В самой точке~$b_1$
они могут быть либо меньше~$a_1$, либо равны~$a_1$. Любая
функция первого типа меньше любой функции второго типа. Функции
первого типа могут принимать любые значения в точках, меньших~$b_1$, а
в точке~$b_1$ имеют значение из~$[0,a_1)$. Тем самым их
можно отождествить с элементами множества~$[[0,b_1)\to A]\hm\times
[0,a_1)$, и при этом отождествлении сохраняется порядок.

Функции второго типа (равные~$a_1$ в точке~$b_1$) снова
разбиваются на две категории: те, которые в точке~$b_2$ меньше $a_2$ и
те, которые в~$b_2$ равны $a_2$. Функции первой категории
отождествляются с элементами множества
$[[0,b_2)\to A]\hm\times [0,a_2)$.
Функции второй категории снова разобьём на части в зависимости
от их значения в точке~$b_3$~\итд Таким образом, $[0,f)$~как
упорядоченное множество изоморфно множеству
\begin{align*}
 [[0,b_1)\to A]\times [0,a_1)&+ [[0,b_2)\to A]\times [0,a_2)+\ldots+\\
  &+[[0,b_k)\to A]\times [0,a_k).
\end{align*}
Переходя к ординалам (начальные отрезки\т это меньшие ординалы),
получаем такое утверждение:

\begin{theorem}
        %
Всякий ординал, меньший $\alpha^\beta$, представляется в
виде
        $$
\alpha^{\beta_1}\alpha_1 + \alpha^{\beta_2}\alpha_2+\ldots+
\alpha^{\beta_k}\alpha_k,
        $$
где $\beta\hm>\beta_1\hm>\beta_2\hm>\ldots\hm>\beta_k$,
а $\alpha_1,\alpha_2,\dots,\alpha_k\hm<\alpha$.
Такое представление однозначно и любая сумма указанного вида
является ординалом, меньшим~$\alpha^\beta$.
        %
\end{theorem}

\begin{proof}
        %
Возможность такого представления мы уже доказали. Последнее
утверждение следует из того, что любая сумма такого вида
является начальным отрезком в множестве~$[B\to A]$ (где $A$
и~$B$ упорядочены по типам $\alpha$ и~$\beta$) и разным
суммам соответствуют разные начальные отрезки.
        %
\end{proof}

Это утверждение обобщает описанную нами ранее
(с.~\pageref{position-ordinal-system}) \лк позиционную систему%
\index{Позиционная система счисления}
обозначений с основанием $\alpha$\пк\ для ординалов, меньших
$\alpha^k$; теперь вместо~$k$ можно использовать любой ординал.

Можно было бы сразу сказать, что элементами
$[B\to A]$
являются формальные суммы вида
        $$
\alpha^{\beta_1}\alpha_1 + \alpha^{\beta_2}\alpha_2+\ldots+
\alpha^{\beta_k}\alpha_k
        $$
(где $\beta\hm>\beta_1\hm>\ldots\hm>\beta_k$ и
$\alpha_1,\dots,\alpha_k\hm<\alpha$) с естественным порядком на них.

Теперь уже понятно, как устроены ординалы в последовательности
        $$
\omega^\omega, \omega^{(\omega^\omega)},\ldots
        $$
Первый из них образован \лк одноэтажными\пк\
выражениями вида
        $$
\omega^{b_1}a_1 + \omega^{b_2}a_2+\ldots+
\omega^{b_k}a_k,
        $$
где $a_i$ и $b_i$\т натуральные числа (и $b_1\hm>\dots\hm>b_k$).
Если в качестве
$b_1, \dots, b_k$ разрешить писать любые \лк одноэтажные\пк\
выражения указанного вида, то полученные \лк двухэтажные\пк\ выражения
упорядочены по типу~$\omega^{(\omega^\omega)}$. Разрешив в показателях
двухэтажные выражения, мы получим трехэтажные выражения, которые образуют
следующий ординал~\итд  Если объединить все эти множества, то есть не
ограничивать число этажей (которое для каждого выражения тем не
менее конечно), то получится множество, упорядоченное по типу
        $$
\sup(\omega, \omega^\omega, \omega^{(\omega^\omega)},\ldots)
        $$
Этот ординал обозначается~$\varepsilon_0$.

\begin{problem}
        %
Докажите, что
        $$
\varepsilon_0 = \omega + \omega^\omega + \omega^{(\omega^\omega)}+\ldots
        $$
\end{problem}

\begin{problem}
        %
Определим для натуральных чисел операцию \лк тотальной замены основания~$k$
на~$l$\пк\ (здесь $k$ и~$l$\т натуральные числа, причём~$l\hm>k$)
следующим образом: данное число~$n$ запишем в $k$\д ичной системе, то есть
разложим по степеням~$k$, показатели степеней снова запишем в $k$\д ичной
системе, новые показатели также разложим~\итд Затем на всех уровнях
заменим основание~$k$ на основание~$l$ и вычислим значение получившегося
выражения. Докажите, что начав с любого~$n$ и выполняя последовательность
операций \лк вычитание
единицы~-- тотальная замена основания~$2$ на~$3$~--
вычитание единицы~-- тотальная замена основания~$3$ на~$4$~--
вычитание единицы~-- тотальная замена основания~$4$ на~$5$~--\,$\dots$\пк,
мы рано или поздно зайдём в тупик, \те
получится нуль и вычесть единицу будет
нельзя. (Указание: заменим все основания сразу на ординал~$\omega$; получится
убывающая последовательность ординалов, меньших~$\varepsilon_0$.)
        %
\end{problem}

\section{Приложения ординалов}

В большинстве случаев рассуждения с использованием трансфинитной
индукции и ординалов можно заменить ссылкой на
лемму~Цорна\index{Лемма Цорна}; при этом рассуждение становится
менее наглядным, но формально более простым. Тем не менее бывают
ситуации, когда такая замена затруднительна. В этом разделе мы
приведём два подобных примера.

Первый из них касается борелевских множеств\index{Борелевское множество}%
\index{Множества!борелевские}. (Для простоты мы
        \glossary{Борель@\Борель}%
рассматриваем подмножества действительной прямой.) Семейство
подмножеств называется
$\sigma$\д\emph{алгеброй}\index{Алгебра@$\sigma$\д алгебра}, если
оно замкнуто относительно
конечных и счётных пересечений и объединений, а также
относительно перехода к дополнению. (Это означает, что вместе с
каждым множеством~$A$ это семейство содержит его
дополнение~$\bbR\setminus A$, и вместе с любыми множествами
$A_0$, $A_1$, $\dots$ семейство содержит их объединение $A_0\hm\cup
A_1\hm\cup\ldots$ и пересечение $A_0\hm\cap A_1\hm\cap\ldots$) Пример:
семейство~$P(\bbR)$ всех подмножеств прямой, очевидно, является
$\sigma$\д алгеброй.

\begin{theorem}
        \label{borel-sets}
Существует наименьшая $\sigma$\д алгебра, содержащая
все отрезки~$[a,b]$ на прямой.
        %
\end{theorem}

\begin{proof}
        %
Формально можно рассуждать так: рассмотрим все возможные
$\sigma$\д алгебры, содержащие отрезки. Их пересечение будет
$\sigma$\д алгеброй, и тоже будет содержать все отрезки. (Вообще
пересечение любого семейства $\sigma$\д алгебр будет $\sigma$\д
алгеброй\т это очевидное следствие определения.) Эта $\sigma$\д
алгебра и будет искомой.
        %
\end{proof}

Множества, входящие в эту наименьшую $\sigma$\д алгебру,
называют \emph{борелевскими}\index{Борелевское множество}%
\index{Множества!борелевские}.

\begin{problem}
        %
Докажите, что всякое открытое и всякое замкнутое подмножество
прямой является борелевским. (Указание: открытое множество
есть объединение содержащихся в нём отрезков с рациональными
концами.)
        %
\end{problem}

\begin{problem}
        %
Докажите, что прообраз\index{Прообраз} любого
борелевского множества при непрерывном
отображении является борелевским множеством.
        %
\end{problem}

\begin{problem}
        %
Пусть $f_0$, $f_1$, $\dots$\т последовательность непрерывных функций
с действительными аргументами и значениями. Докажите, что множество
точек~$x$, в которых последовательность $f_0(x)$, $f_1(x)$, $\dots$
имеет предел, является борелевским.
        %
\end{problem}

\problskip
Борелевские множества играют важную роль в \emph{дескриптивной
теории множеств}\index{Дескриптивная теория множеств}. Но мы
хотим лишь продемонстрировать использование трансфинитной
индукции (которую трудно заменить на лемму~Цорна) на примере
следующей теоремы:

\begin{theorem}
        \label{borel-sets-continuum}%
        %
Семейство всех борелевских множеств\index{Борелевское множество}%
\index{Множества!борелевские} имеет мощность континуума.
        %
\end{theorem}

\begin{proof}
        %
Класс борелевских множеств можно строить постепенно. Начнём с
отрезков и дополнений к отрезкам. На следующем шаге рассмотрим
всевозможные счётные пересечения и объединения уже построенных
множеств (отрезков и дополнений к ним).

\begin{problem}
        %
Докажите, что при этом получатся (среди прочего) все открытые
и все замкнутые подмножества прямой.
        %
\end{problem}

\problskip
Далее можно рассмотреть счётные объединения и пересечения
уже построенных множеств~\итд

Более формально, пусть $\mathcal{B}_0$\т семейство множеств,
состоящее из всех отрезков и дополнений к ним. Определим~$\mathcal{B}_{i+1}$
по индукции как семейство множеств, являющихся
счётными объединениями или пересечениями множеств из~$\mathcal{B}_i$.

Все семейства~$\mathcal{B}_i$ состоят из борелевских множеств
(поскольку счётное объединение или пересечение борелевских
множеств является борелевским). Исчерпывают ли они все
борелевские множества? Вообще говоря, нет: если мы возьмём по
одному множеству из каждого класса~$\mathcal{B}_i$
для всех $i=0,1,2,\dots$ и рассмотрим их
счётное пересечение, то оно вполне может не принадлежать ни
одному из классов. Поэтому мы рассмотрим класс~$\mathcal{B}_\omega$,
представляющий собой объединение всех~$\mathcal{B}_i$ по всем
натуральным~$i$, затем~$\mathcal{B}_{\omega+1}$,
$\mathcal{B}_{\omega+2}$~\итд Объединение этой последовательности
классов естественно
назвать~$\mathcal{B}_{\omega2}$ и продолжить построение.

Дадим формальное определение~$\mathcal{B}_\alpha$%
\index{$\mathcal{B}_\alpha$}
для любого ординала~$\alpha$. Это делается с
помощью трансфинитной рекурсии. При~$\alpha\hm=\beta\hm+1$
элементами класса~$\mathcal{B}_{\alpha}$ будут счётные объединения и
пересечения множеств из класса~$\mathcal{B}_\beta$. Если $\alpha$\т
предельный ординал, отличный от~$0$, то класс~$\mathcal{B}_\alpha$
представляет собой объединение всех~$\mathcal{B}_\beta$ по
всем~$\beta\hm<\alpha$. (Класс~$\mathcal{B}_0$ мы уже определили.)

Из определения следует, что $\mathcal{B}_\alpha \hm\subset
\mathcal{B}_\beta$ при~$\alpha\hm<\beta$, так что мы получаем
возрастающую цепь классов. Каждый класс замкнут относительно
перехода к дополнению (для начального класса мы об этом
позаботились, далее по индукции). Все классы~$\mathcal{B}_{\alpha}$
содержатся в классе борелевских множеств, так как мы применяем
лишь операции счётного объединения и пересечения, относительно
которых класс борелевских множеств замкнут.

Возникает вопрос: как далеко нужно продолжать эту конструкцию?
Оказывается, что достаточно дойти до первого несчётного
ординала.

Пусть $\aleph_1$\index{$\aleph_1$}\т наименьший
несчётный ординал. (Это\т стандартное
для него обозначение.) Другими словами, $\aleph_1$ есть семейство всех
счётных ординалов, упорядоченных отношением $<$ на ординалах.

\textsf{Лемма.} Класс $\mathcal{B}_{\aleph_1}$ замкнут
относительно счётных объединений
и пересечений и потому содержит все борелевские множества.

Доказательство леммы. Пусть имеется счётная последовательность
множеств $B_0, B_1, \dots$, принадлежащих $\mathcal{B}_{\aleph_1}$.
Ординал~$\aleph_1$\т предельный, и класс~$\mathcal{B}_{\aleph_1}$
является объединением
меньших классов. Поэтому каждое из множеств~$B_i$ принадлежит
какому\д то классу~$\mathcal{B}_{\alpha_i}$, где $\alpha_i$\т
некоторый ординал, меньший~$\aleph_1$, \те конечный или
счётный ординал. Положим $\beta=\sup_i \alpha_i$. Ординал~$\beta$ есть
точная верхняя грань счётного числа счётных
ординалов и потому счётен. В самом деле, рассмотрим
ординалы~$\alpha_i$ как начальные отрезки в каком\д то большем ординале
(например, в~$\aleph_1$); их точная верхняя грань будет
объединением счётного числа счётных начальных отрезков и
потому будет счётным ординалом.

Теперь первое утверждение леммы очевидно: все~$B_i$ лежат
в~$\mathcal{B}_\beta$, а потому их объединение (или пересечение) лежит
в~$\mathcal{B}_{\beta+1}$ и тем более в~$\mathcal{B}_{\aleph_1}$
(поскольку $\beta\hm+1$~есть счётный ординал и меньше~$\aleph_1$).

Таким образом, класс $\mathcal{B}_{\aleph_1}$ является $\sigma$\д
алгеброй, содержащей отрезки, и потому содержит все
борелевские множества. Лемма доказана.

Как мы уже отмечали, все классы~$\mathcal{B}_{\alpha}$ состоят
из борелевских множеств, так что класс~$\mathcal{B}_{\aleph_1}$
совпадает с классом всех борелевских множеств.

Что можно сказать про мощность классов? Класс~$\mathcal{B}_0$ имеет
мощность континуума (отрезки задаются своими концами). Если
класс~$\mathcal{B}_\alpha$ имеет мощность континуума, то и следующий
класс~$\mathcal{B}_{\alpha+1}$ имеет мощность континуума (каждый его
элемент задаётся счётной последовательностью элементов
предшествующего класса, а~$\mathfrak{c}^{\aleph_0}\hm=\mathfrak{c}$).
Каждый предельный класс
есть объединение предыдущих, и пока мы не выходим за пределы
счётных ординалов, объединение это будет счётно,
а~$\mathfrak{c}\aleph_0\hm=\mathfrak{c}$, так что мы не выходим за
пределы континуума. Наконец, $\mathcal{B}_{\aleph_1}$~есть
объединение несчётного числа предыдущих классов (а именно,
$\aleph_1$~классов), но так как $\aleph_1\hm\le\mathfrak{c}$,
то~$\mathfrak{c}\aleph_1\hm=\mathfrak{c}$.

Таким образом, класс~$\mathcal{B}_{\aleph_1}$, он же класс всех
борелевских множеств, имеет мощность континуума.
        %
\end{proof}

Обычно построение борелевских множеств начинается немного иначе.
Именно, на нижнем уровне рассматриваются два класса: открытые\index{Открытое
множество}\index{Множества!открытые} и
замкнутые\index{Замкнутое множество}\index{Множества!замкнутые} множества.
На следующем уровне находятся
классы~$F_\sigma$\index{$F_\sigma$} (счётные объединения замкнутых множеств)
и~$G_{\delta}$\index{$G_{\delta}$} (счётные пересечения открытых множеств). Ещё на
уровень выше лежат счётные пересечения множеств из~$F_{\sigma}$ и счётные
объединения множеств
из~$G_{\delta}$,~\итд Такой подход является
более естественным
с точки зрения топологии, поскольку отрезки на прямой ничем не
замечательны. Можно проверить, что разница между таким подходом
и нашим определением невелика.

\problskip
\begin{problem}
        %
Докажите, что пересечение двух $F_{\sigma}$\д множеств является
$F_{\sigma}$\д множеством (и вообще классы $F_{\sigma}$, $G_{\delta}$,
а также классы следующих уровней, замкнуты
относительно конечных объединений и пересечений).
        %
\end{problem}

\begin{problem}
        %
Докажите, что $F_{\sigma}$- и $G_{\delta}$\д множества лежат
в классе~$\mathcal{B}_2$ в соответствии с нашей классификацией.
        %
\end{problem}

\begin{problem}
        %
Докажите, что всякое множество класса~$\mathcal{B}_2$ отличается от
некоторого $F_{\sigma}$- или $G_{\delta}$\д множества не более
чем на счётное множество.
        %
\end{problem}

\begin{problem}
        %
Докажите, что всякое множество класса~$\mathcal{B}_3$ является
счётным пересечением $F_{\sigma}$\д множеств или счётным объединением
$G_{\delta}$\д множеств и что аналогичное утверждение верно
для более высоких уровней нашей иерархии.
        %
\end{problem}

\begin{problem}
        %
Докажите, что существует открытое множество на плоскости,
среди вертикальных сечений которого встречаются все
открытые подмножества прямой. Докажите, что существует
$G_{\delta}$\д множество на плоскости, среди сечений которого
встречаются все $G_{\delta}$\д подмножества прямой.
Докажите аналогичные утверждения для следующих уровней.
        %
\end{problem}

\begin{problem}
        %
Покажите, что существует $G_{\delta}$\д множество,
не являющееся $F_{\sigma}$\д множеством. Покажите, что
существует счётное объединение $G_{\delta}$\д множеств,
не являющееся счётным пересечением $F_{\sigma}$\д множеств
\итд (Указание: воспользуйтесь предыдущей задачей.)
        %
\end{problem}

\medskip

Ординалы часто появляются при классификации элементов
того или иного множества по \лк рангам\пк. Например, можно
классифицировать элементы фундированного множества\index{Фундированное
множество}\index{Множества!фундированные}.

\begin{theorem}
        \label{ranks-for-well-founded-sets}
Пусть $X$\т фундированное множество. Тогда существует и
единственна функция~$\rk$, определённая на~$X$ и принимающая
значения в классе ординалов, для которой
        $$
\rk(x)=\min\{\alpha\mid\text{$\alpha>\rk(y)$ для любого $y<x$}\}
        $$
(при любом~$x\hm\in X$).
        %
\end{theorem}

\begin{proof}
        %
Определим множество~$X_{\alpha}$ рекурсией по ординалу~$\alpha$:
$X_{\alpha}$~состоит из всех элементов~$x\hm\in X$, для которых
все меньшие их (в~$X$) элементы принадлежат~$X_{\beta}$ с
меньшими индексами~$\beta$:
        $$
x \in X_{\alpha}
     \ \Leftrightarrow \
(\forall y < x)\,(\exists \beta < \alpha)\, (y\in X_{\beta}).
        $$
Заметим, что здесь (как и в формулировке теоремы) знак~$<$
используется в двух разных смыслах: как порядок на~$X$
и как порядок на ординалах.

Очевидно, что с ростом~$\alpha$ множество~$X_{\alpha}$ растёт
(точнее, не убывает). Докажем, что при достаточно
большом~$\alpha$ множество $X_{\alpha}$ покрывает всё~$X$. Если
это не так, то из~$\beta \hm< \gamma$ следует~$X_{\beta}\hm\subsetneq
X_{\gamma}$ (произвольный минимальный элемент, не лежащий в~$X_{\beta}$,
принадлежит~$X_\gamma$). Поэтому отображение $\alpha\hm\mapsto
X_{\alpha}$ будет инъекцией, что невозможно (возьмём ординал, по
мощности больший~$P(X)$; предшествующих ему ординалов уже
слишком много).

Теперь определим~$\rk(x)$ как минимальное значение~$\alpha$, для
которого~$x\hm\in X_{\alpha}$. Если $\rk(x)=\alpha$ и~$y\hm<x$,
то~$\rk(y)\hm<\alpha$. (В~самом деле, по определению
$X_{\alpha}$ из $x\hm\in X_\alpha$ и~$y\hm<x$ следует,
что~$y\hm\in X_{\beta}$ при некотором~$\beta\hm<\alpha$.)
Наоборот, если для некоторого ординала~$\gamma$ выполнено
неравенство~$\rk(y)\hm<\gamma$ при всех~$y\hm<x$,
то~$\rk(x)\hm\le\gamma$. В~самом деле, тогда любой
элемент~$y\hm<x$ принадлежит некоторому~$X_{\beta}$
с~$\beta\hm<\gamma$ (положим~$\beta\hm=\rk(y)$) и
потому~$x\hm\in X_{\gamma}$ и~$\rk(x)\hm\le\gamma$.

Итак, построенная нами функция~$\rk$ обладает требуемым
свойством. Единственность доказать совсем легко: если есть
две такие функции, рассмотрим минимальную точку в~$X$, на которой
они различаются, и сразу же получим противоречие.
        %
\end{proof}

В частности, счётные ординалы можно использовать для
классификации деревьев\index{Дерево}, в которых нет бесконечных путей. Мы
будем рассматривать
\emph{корневые деревья с конечным или счётным ветвлением}
(у
каждой вершины конечное или счётное число сыновей), в которых
нет бесконечной ветви (последовательности вершин, в которых
каждая есть сын предыдущей).

Формально такое дерево можно определить как подмножество~$T$
множества~$\bbN^*$\index{$\bbN^*$} конечных последовательностей натуральных
чисел, замкнутое относительно взятия префикса (если
последовательность принадлежит~$T$, то любой её начальный
отрезок принадлежит~$T$). Элементы множества~$T$ мы называем
\emph{вершинами} дерева; вершина~$y$ есть
\emph{сын} вершины~$x$, если
$y$~получается из~$x$ приписыванием справа какого\д то одного
числа. Вершина~$y$ является \emph{потомком} вершины~$x$,
если~$y$ получается добавлением к~$x$ одного или нескольких
чисел.

Мы говорим, что в дереве~$T$ \emph{нет бесконечной ветви},
если
не существует бесконечной последовательности натуральных чисел,
все начала которой принадлежат~$T$. В~этом случае отношение
порядка
        $$
y<x \ \Leftrightarrow \ \text{$y$ есть потомок~$x$}
        $$
фундировано и можно применить предыдущую теорему, определив
ранги всех вершин дерева~$T$. Ранг его \emph{корня}
(последовательности длины~$0$) и будем называть \emph{рангом
дерева}\index{Ранг дерева}.

\begin{theorem}
        \label{countable-ordinals}%
(\textsf{а})~%
Ранг любого дерева (описанного вида) является счётным ординалом.

(\textsf{б})~%
Всякий счётный ординал является рангом некоторого дерева.
        %
\end{theorem}

\begin{proof}
        %
(а)~%
Пусть ранг некоторого дерева, то есть ранг его корня, является
несчётным ординалом. Тогда ранг одного из сыновей корня также
несчётен. (В самом деле, точная верхняя грань счётного множества
счётных ординалов является счётным ординалом; это становится
ясным, если рассматривать эти ординалы как начальные отрезки
большего\т тогда точная верхняя грань будет объединением.)
У этого сына в свою очередь есть сын несчётного ранга~\итд
Этот процесс не может оборваться, и мы получаем бесконечную
ветвь в противоречии с предположением.

(б)~%
Это утверждение доказывается индукцией: пусть $\alpha$\т наименьший
счётный ординал, для которого такого дерева нет. Тогда для
всех меньших ординалов деревья
есть. Возьмём эти деревья
и сделаем их поддеревьями с общим корнем (их корни станут
сыновьями этого общего корня). Новое дерево также имеет
счётное ветвление и ранг его корня равен~$\alpha$.
        %
\end{proof}

\problskip
\begin{problem}
        %
Пусть имеется счётное дерево, не имеющее бесконечных ветвей.
Предположим, что в каждом его листе находится отрезок или
дополнение до отрезка, а в каждой внутренней вершине стоит знак
пересечения или объединения. Как сопоставить такому дереву
некоторое борелевское множество? (Указание: покажите, что в
каждой вершине можно единственным образом написать некоторое
множество, согласованное с пометками.) Покажите, что все
борелевские множества могут быть получены таким способом.
        %
\end{problem}

Деревья с пометками, рассмотренные в этой задаче, представляют
собой как бы бесконечные формулы, составленные из отрезков
и дополнений к ним с помощью операций счётного объединения
и пересечения (конечные деревья соответствовали бы
конечным формулам). Можно условно сказать, что борелевские
множества\т это множества, выражающиеся с помощью таких
формул.

\begin{problem}
        %
Докажите, что семейство борелевских множеств имеет мощность
континнума, используя \лк бесконечные формулы\пк\т размеченные
деревья, в которых нет бесконечных ветвей. (Это доказательство
обходится без ординалов, трансфинитной индукции и даже леммы
Цорна\т хотя и использует аксиому выбора.)
        %
\end{problem}

\medskip

В заключение приведём скорее забавный, чем важный, пример
использования трансфинитной рекурсии и ординалов.

\begin{theorem}
        \label{funny-set}
Существует множество точек на плоскости, которое
пересекается с каждой прямой ровно в двух точках.
        %
\end{theorem}

Две параллельные прямые почти что удовлетворяют этому требованию
(исключением являются лишь параллельные им прямые). Но
избавиться от этого исключения не так просто.

\begin{proof}
        %
Требования к множеству можно сформулировать так: никакие
три точки не лежат на одной прямой, но любая прямая
пересекает его не менее чем в двух точках.

Будем строить это множество трансфинитной рекурсией. Пусть
$\alpha$\т минимальный ординал, имеющий мощность континуума.
(Если континуум\д гипотеза\index{Гипотеза континуума} верна,
то он совпадает с $\aleph_1$,
но это нам не важно.) Тогда множество всех меньших ординалов
можно поставить во взаимно однозначное соответствие с
множеством всех прямых на плоскости. Пусть $l_{\beta}$\т прямая,
соответствующая ординалу~$\beta\hm<\alpha$.

Для каждого~$\beta\hm<\alpha$ построим множество~$M_{\beta}$, в
котором никакие три точки не лежат на одной прямой, следующим
образом. Объединим все построенные ранее множества~$M_{\gamma}$
при всех~$\gamma\hm<\beta$. Могут ли в этом множестве (обозначим
его~$T$) какие\д то три точки лежать на одной прямой? Если да,
то эти точки берутся из каких\д то множеств $M_{\gamma_1}$, $M_{\gamma_2}$,
$M_{\gamma_3}$; возьмём наибольший из ординалов
$\gamma_1$, $\gamma_2$, $\gamma_3$; в соответствующем множестве будут
три точки, лежащие на одной прямой, что противоречит
предположению индукции.

Посмотрим, во скольких точках пересекает прямая $l_{\beta}$
множество~$T$. Таких точек (по доказанному) не больше двух.
Если их ровно две, то всё хорошо и мы новых точек не добавляем,
считая, что~$M_{\beta}\hm=T$. Если их меньше, то мы должны
добавить новые точки (до двух), но только так, чтобы
при этом не образовалось трёх точек, лежащих на одной прямой.
Другими словами, нельзя добавлять точки, которые лежат на
пересечении~$l_{\beta}$ с прямыми, проходящими через пары
уже имеющихся точек.

Сколько таких прямых (то есть сколько пар уже имеющихся точек)?
По построению видно, что все уже имеющиеся точки лежат по две на
каждой прямой~$l_{\gamma}$ при~$\gamma\hm<\beta$. (Строго говоря,
это следует включить в индуктивное предположение.) Таким
образом, множество~$T$ по мощности есть~$2\beta\hm=\beta$, а пар
точек не больше $\beta^2\hm=\beta\hm<\mathfrak{c}$. Поэтому
запрещённые точки
составляют лишь малую (по мощности) часть прямой~$l_{\beta}$, и
можно выбрать две разрешённые точки.

Теперь осталось объединить множества~$M_{\beta}$ для всех
ординалов~$\beta\hm<\alpha$ и получить искомое множество. (По
условию три точки на одной прямой в нём появиться не могут, а
всякая прямая будет рано или поздно рассмотрена и две точки на
ней будут обеспечены.)
        %
\end{proof}

\begin{problem}
        %
Найдите ошибку в следующем \лк опровержении гипотезы
континуума\пк:\index{Континуум\д гипотеза} пусть
$\aleph_1=\mathfrak{c}$. Упорядочим отрезок $[0,1]$ по
типу~$\aleph_1$. Рассмотрим функцию двух переменных, равную
единице на паре $(x,y)$, если $x<y$ (в смысле этого порядка), и
нулю в остальных случаях. Тогда при фиксированном $x$ функия
$y\mapsto f(x,y)$ равна единице везде, кроме счётного
множества, и потому интегрируема и $\int f(x,y)\,dy=1$ при
любом~$x$. С другой стороны, функция $x\mapsto f(x,y)$ равна
нулю всюду, кроме счётного множества, так что $\int
f(x,y)\,dx=0$. Получаем противоречие с теоремой
Фубини\index{Фубини теорема}\index{Теорема!Фубини}\glossary{Фубини@\Фубини}, которая
утверждает, что
        $$
\int_0^1\left(\int_0^1 f(x,y)\,dy\right)\,dx=
\int_0^1\left(\int_0^1 f(x,y)\,dx\right)\,dy.
        $$
\end{problem}
